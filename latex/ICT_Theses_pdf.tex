%%%%%%%%%%%%%%%%%%%%%%%%%%%%%%%%%%%%%%%%%%%%%%%%%%%%%%%%%%%%%%%%%%%%%%%
%                                                                     %
%   Institut f�r Computertechnik, Gusshausstra�e 27-29, 1040 Wien     %
%                                                                     %
% Basisdokument f�r wissenschaftliche Arbeiten am ICT im pdf-    			%
% Format; entsprechenden Schalter unten setzen (\isDiss, etc.)        %
%                                                                     %
%	Basiert auf der Dokumentklasse report.                        			%
%                                                                     %
% Autor: Dietmar Bruckner                                             %
% Version: $Revision: 1.2 $                                           %
% Datum: $Date: 2009/11/25 $				                                  %
%                                                                     %
% --- ICT_Theses_pdf.tex ---  	                                      %
%                                                                     %
%%%%%%%%%%%%%%%%%%%%%%%%%%%%%%%%%%%%%%%%%%%%%%%%%%%%%%%%%%%%%%%%%%%%%%%
% Some options of this template requres the usage of pdfltex.         %
% For the ease of use, please call the makepdf.bat from skripte dir   % 
%%%%%%%%%%%%%%%%%%%%%%%%%%%%%%%%%%%%%%%%%%%%%%%%%%%%%%%%%%%%%%%%%%%%%%%

%%%%%%%%%%%%%%%%%%%%%%%%%%%%%%%%%%%%%%%%%%%%%%%%%%
% Main document for dissertation
%%%%%%%%%%%%%%%%%%%%%%%%%%%%%%%%%%%%%%%%%%%%%%%%%%

%%%%%%%%%%%%%%%%%%%%%%%%%%%%%%%%%%%%%%%%%%%%%%%%%%
% 1.PACKAGES
%%%%%%%%%%%%%%%%%%%%%%%%%%%%%%%%%%%%%%%%%%%%%%%%%%

% defines thesis as report (oneside, A4 with 11pt fontsize)
\documentclass[11pt,twoside,a4paper]{ICTthesis}
% formats the text accourding the set language
\usepackage[english]{babel}
% generates indices with the "\index" command
\usepackage{makeidx}
% enables import of graphics. We use pdflatex here so do the pdf optimisation.
%\usepackage[dvips]{graphicx}
\usepackage[pdftex]{graphicx}
\usepackage{pdfpages}
% includes floating objects like tables and figures.
\usepackage{float}
% for generating subfigures with ohne indented captions
\usepackage[hang]{subfigure}
% redefines and smartens captions of figures and tables (indentation, smaller and boldface)
\usepackage[hang,small,bf]{caption}
% enables tabstops and the numeration of lines
\usepackage{moreverb}
% enables user defined header and footer lines (former "fancyheadings")
\usepackage{fancyhdr}
% Some smart mathematical stuff
\usepackage{amsmath}
% Package for rotating several objects
\usepackage{rotating}
\usepackage{natbib}
\usepackage{epsf}
\usepackage{dsfont}
% \usepackage[usenames]{color}
\usepackage[algochapter, boxruled, vlined]{algorithm2e}
%Activating and setting of character protruding - if you like
%\usepackage[activate,DVIoutput]{pdfcprot}
% If you really need special chars...
\usepackage[latin1]{inputenc}
% Hyperlinks
\usepackage[colorlinks,hyperindex,plainpages=false,%
pdftitle={Master thesis: Sample Title},%
pdfauthor={Authors name},%
pdfsubject={Master thesis},%
pdfkeywords={keyword},%
pdfpagelabels,%
pagebackref,%
bookmarksopen=false%
]{hyperref}
% For the two different reference lists ...
\usepackage{multibib}
\usepackage{multicol}


%%%%%%%%%%%%%%%%%%%%%%%%%%%%%%%%%%%%%%%%%%%%%%%%%%
% 2.Settings
%%%%%%%%%%%%%%%%%%%%%%%%%%%%%%%%%%%%%%%%%%%%%%%%%%

\newif\ifisDiss
\newif\ifisDipE
\newif\ifisDipD
\newif\ifisBach

%VERTIEFUNG/PRAKTIKUM/REFERAT: set all to false;
%and correct respective lines in "`Titlepage_all.tex"'

\isDissfalse
\isDipEfalse
\isDipDfalse
\isBachtrue

\ifisDiss
\newcites{weblink}{Internet References}
\else
\ifisDipE
\newcites{weblink}{Internet References}
\else
\ifisDipD
\newcites{weblink}{Internet Referenzen}
\else
\ifisBach
\newcites{weblink}{Internet Referenzen}
\else
\newcites{weblink}{Internet Referenzen}
\fi
\fi
\fi
\fi
% redifine the paragraph command.
\makeatletter
\renewcommand\paragraph{\@startsection{paragraph}{4}{\z@}%
                                    {3.25ex \@plus1ex \@minus.2ex}%
                                    {0.3em} %-1em}%
                                    {\normalfont\normalsize\bfseries}}

\renewcommand\subparagraph{\@startsection{subparagraph}{5}{\parindent}%
                                       {3.25ex \@plus1ex \@minus .2ex}%
                                       {-1em}%
                                      {\normalfont\normalsize\bfseries}}
\makeatother

%Enables numbers at subsubsections without inserting them into the toc.
\setcounter{secnumdepth}{3}

% generates the index (command for the subprocessor)
\makeindex

% default path to your pictures
\graphicspath{{pictures/}}

% Counter for the maximum number of "Floatobjects" at the beginning of the page.
\setcounter{topnumber}{2}
% Redefines the maximum area which floats my consume at the beginning of the page.
\def\topfraction{.8}
% Counter for the maximum numbers of floats at the end of the page
\setcounter{bottomnumber}{2}
% Redefines the maximum area which floats my consume at the end of the page.
\def\bottomfraction{.5}
% Maximal number of floats per page
\setcounter{totalnumber}{8}
% minimal amount of text per page
\def\textfraction{.2}
% Redefinition: minimal amount of floats in percent per floatpage.
\def\floatpagefraction{.6}
% no indentation at paragraphs
\setlength{\parindent}{0pt}

% part of the caption package: extra 20pts left and right of captions.
%\setlength{\captionmargin}{20pt}

% sets the page layout
\setlength{\oddsidemargin}{4mm}
\setlength{\evensidemargin}{-6mm}
\setlength{\textwidth}{162mm} 
\setlength{\textheight}{230mm}
\setlength{\topmargin}{-5mm}
%\addtolength{\headsep}{12pt}

%part of the "float" Packages:
\floatstyle{plain}
% define a new floating object
\floatname{example}{Example}

\newfloat{example}{hbtp}{loe}[chapter]
\floatplacement{figure}{hbt}
\floatplacement{table}{htb}

% enables a "\dollar" command (returns $)
\newcommand{\dollar}{\char36}

% Script for abbreviations
% defines a new environment with one arguement
\newenvironment{bfscript}[1] {
 % defines as list
 \begin{list}
 % No labelmarks!
 {}
 {\settowidth{\labelwidth}{\small #1}
  % sets the left margin to 0 because there is no labelmark
  \setlength{\leftmargin}{\labelwidth}
  % add labelsep (0, no labelmark) to the margin
  \addtolength{\leftmargin}{\labelsep}
  % Separation of paragraphs in one topic
  \parsep 0.0ex plus 0.2ex minus 0.2ex
  % Separation of two topics
  \itemsep -0.3ex
  % sets the label to: small and fills with whitspace to the text
  \renewcommand{\makelabel}[1]{\small ##1\hfill}}}
 {\end{list}
}

% PDF-Settings
\def\pdfBorderAttrs{/Border [0 0 0] } % No border arround Links
\pdfcompresslevel=9
\hypersetup{colorlinks,linkcolor=blue,filecolor=red,urlcolor=black,citecolor=blue}

%%%%%%%%%%%%%%%%%%%%%%%%%%%%%%%%%%%%%%%%%%%%%%%%%%
% 3.HYPENATION
%%%%%%%%%%%%%%%%%%%%%%%%%%%%%%%%%%%%%%%%%%%%%%%%%%

% enter special rules here!
\hyphenation{gleich-zeitig para-meter}

%%%%%%%%%%%%%%%%%%%%%%%%%%%%%%%%%%%%%%%%%%%%%%%%%
% 4.Begin of the real document
%%%%%%%%%%%%%%%%%%%%%%%%%%%%%%%%%%%%%%%%%%%%%%%%%%

\begin{document}
% reads the new commands


\newcommand{\mat}[1]{\ensuremath{\mathbf #1}}
\newcommand{\set}[1]{\ensuremath{\mathbf #1}}
\newcommand{\cset}[1]{\ensuremath{\mathbf{\mathcal #1}}}
\renewcommand{\vec}[1]{\ensuremath{\mathbf #1}}

\newcommand{\nth}[1]{\ensuremath{#1^\mathrm{th}}}
\newcommand{\fst}[1]{\ensuremath{#1^\mathrm{st}}}
\newcommand{\snd}[1]{\ensuremath{#1^\mathrm{nd}}}

\newcommand{\argmax}[1]{\ensuremath{\arg\hspace{-0.4ex}\max_{\hspace*{-3.0ex}#1}}}
\newcommand{\argmin}[1]{\ensuremath{\arg\min_{\hspace*{-4.0ex}#1}}}

\newcommand{\argmaxi}[1]{\ensuremath{\arg\hspace{-0.4ex}\max_{#1}}}
\newcommand{\argmini}[1]{\ensuremath{\arg\hspace{-0.4ex}\min_{#1}}}

% \newcommand{\argmin}[1]{\ensuremath{\begin{array}[t]{c} \arg \min \\
% \vspace*{-0.1ex} #1 \end{array}}}

\newcommand{\NP}{\ensuremath{\mathcal{NP}}}
\newcommand{\PP}{\ensuremath{\mathcal{P}}}
\newcommand{\e}[2]{\ensuremath{\{#1,#2\}}}
\newcommand{\tup}[1]{\ensuremath{\langle#1\rangle}}
\newcommand{\bigO}[1]{\ensuremath{\mathcal{O}\left(#1\right)}}

\newcommand{\trans}[1]{\ensuremath{{#1}^\top}}
\newcommand{\diag}[1]{\ensuremath{\mathrm{diag}\left(#1\right)}}

\newcommand{\eq}[1]{equation \ref{#1}}
\newcommand{\Eq}[1]{equation \ref{#1}}
\newcommand{\fig}[1]{figure \ref{#1}}
\newcommand{\Fig}[1]{figure \ref{#1}}
\newcommand{\chap}[1]{chapter \ref{#1}}
\newcommand{\Chap}[1]{chapter \ref{#1}}
\newcommand{\sect}[1]{section \ref{#1}}
\newcommand{\Sect}[1]{section \ref{#1}}

\newcommand{\bydefn}{\ensuremath{\stackrel{\bigtriangleup}{=}}}
\newcommand{\elmat}[2]{\ensuremath{#1 \odot #2}}

\newcommand{\prune}[1]{\ensuremath{\mathrm{prune}\left(#1\right)}}

\newcommand{\labelfig}[2]{\parbox[b]{0.2in}{\Large#1\normalsize\vspace{#2}}}
% \newcommand{\labelfig}[1]{\parbox[b]{0.2in}{#1\vspace{1.8in}}}

% \newcommand{\emptyset}{\ensuremath{\O}}

\renewcommand{\Re}{\mathbb{R}}

\newcommand{\incfig}[3]{\ifx\pdfoutput\undefined
                          \epsfig{#1.eps,#2,#3}
                        \else
                          \epsfig{#1.eps,#2,#3}
                        \fi}
          
                         

% inserts the Titlepage
\pdfbookmark{Titlepage}{title}
%%%%%%%%%%%%%%%%%%%%%%%%%%%%%%%%%%%%%%%%%%%%%%%%%%%%%%%%%%%%%%%%%%%%%%%
%                                                                     %
%   Institut f�r Computertechnik, Gusshausstra�e 27-29, 1040 Wien     %
%                                                                     %
% Basisdokument f�r wissenschaftliche Arbeiten am ICT im pdf-   			%
% Format                                                              %
%                                                                     %
%	Basiert auf die Dokumentklasse report.                        			%
%                                                                     %
% Autor: Dietmar Bruckner                                             %
% Version:                                                            %
% Datum: 28. Aug. 2005                                                %
%                                                                     %
% --- Titelseite.tex ---                                              % 
%                                                                     %
%%%%%%%%%%%%%%%%%%%%%%%%%%%%%%%%%%%%%%%%%%%%%%%%%%%%%%%%%%%%%%%%%%%%%%%
% This document defines the normalized preface of the document.       %
% Please adjust only:                                                 %
% *) any names                                                        %
% *) Abstract                                                         %
% *) Kurzfassung                                                      %
% *) Acknowledgement                                                  %
% *) possible Abbreviations                                           %
%%%%%%%%%%%%%%%%%%%%%%%%%%%%%%%%%%%%%%%%%%%%%%%%%%%%%%%%%%%%%%%%%%%%%%%

\pagestyle{empty}
\pagenumbering{Roman}
\begin{titlepage}
\large
\begin{center}
\ifisDiss
DISSERTATION\\\vfill
{\LARGE\bf Your Title}
\\
[5mm]
Your Subtitle
\else
\ifisDipE
DIPLOMA THESIS\\\vfill
{\LARGE\bf Your Title}
\\
[5mm]
Your Subtitle
\else
\ifisDipD
DIPLOMARBEIT\\\vfill
{\LARGE\bf Titel}
\\
[5mm]
ev. Untertitel
\else
\ifisBach
BACHELORARBEIT\\\vfill
{\LARGE\bf Titel}
\\
[5mm]
ev. Untertitel
\else
VERTIEFUNG/PRAKTIKUM/REFERAT\\\vfill
{\LARGE\bf Titel}
\fi
\fi
\fi
\fi
\\
\vfill
\ifisDiss
Submitted at the
Faculty of Electrical Engineering and Information Technology, TU Wien, Vienna
in partial fulfillment of the requirements for the degree of\\
Doktor der technischen Wissenschaften (equals Ph.D.)\\
\vfill
under supervision of\\\vfill
Full name and degrees of your Supervisor\\
Institut number: 384\\
Institute of Computer Technology\\\vfill
and\\\vfill
Full name and degrees of your Supervisor\\
Institut number: 384\\
Institute of Computer Technology\\\vfill
by\\\vfill
Your full name, Firstname first, then surname\\
Matr.Nr. xxxxxxx\\
single lined adress: street number, zip city\\\vfill
\end{center}
Date\hfill\hrulefill
\else
\ifisDipE
Submitted at the
Faculty of Electrical Engineering and Information Technology, Vienna University of Technology\\
in partial fulfillment of the requirements for the degree of\\
Diplom-Ingenieur (equals Master of Sciences) \\
\vfill
under supervision of\\\vfill
Full name and degrees of your Supervisor Prof.\\
Full name and degrees of your Supervisor\\\vfill
by\\\vfill
Your full name, Firstname first, then surname\\
Matr.Nr. xxxxxxx\\
single lined adress: street number, zip city\\\vfill
\end{center}
Date\hfill\hrulefill
\else
\ifisDipD
ausgef�hrt zur Erlangung des akademischen Grades \\
eines Diplom-Ingenieurs unter der Leitung von\\
\vfill
Name und akad. Grad des Prof.\\
Name und akad. Grad des Betreuers\\\vfill
am\\\vfill
{\Large\bf Institut f�r Computertechnik (E384)}\\
der Technischen Universit�t Wien
\vfill
durch
\vfill
Name\\
Matr.Nr. xxxxxxx\\
Wohnadresse\\\vfill
\end{center}
Wien, am xxx\hfill\hrulefill
\else
\ifisBach
ausgef�hrt zur Erlangung des akademischen Grades \\
eines Bachelors unter der Leitung von\\
\vfill
Name und akad. Grad des Prof.\\
Name und akad. Grad des Betreuers\\\vfill
am\\\vfill
{\Large\bf Institut f�r Computertechnik (E384)}\\
der Technischen Universit�t Wien
\vfill
durch
\vfill
Name\\
Matr.Nr. xxxxxxx\\
Wohnadresse\\\vfill
\end{center}
Wien, am xxx\hfill\hrulefill
\else
ausgef�hrt als Projektdokumentation/Referatsarbeit/Praktikum unter der Leitung von\\
\vfill
Name und akad. Grad des Prof.\\
Name und akad. Grad des Betreuers\\\vfill
am\\\vfill
{\Large\bf Institut f�r Computertechnik (E384)}\\
der Technischen Universit�t Wien
\vfill
durch
\vfill
Name\\
Matr.Nr. xxxxxxx\\
Wohnadresse\\\vfill
\end{center}
Wien, am xxx\hfill\hrulefill
\fi
\fi
\fi
\fi
\end{titlepage}
%

\newpage
\pagestyle{plain}
\null\vfil
\begin{center}\bf Kurzfassung\end{center}
Deutsche Kurzfassung\dots\\
\par\vfil
\ifisDiss
\newpage
\fi
\null\vfil
\begin{center}\bf Abstract\end{center}
short version of your thesis \dots
\par\vfil\null
%
\ifisDiss
\newpage
\null\vfil
\begin{center}\bf Vorwort/Preface\end{center}
\par\vfil\null
%
%
\fi
\newpage
\null\vfil
\ifisDiss
\begin{center}\bf Acknowledgements (optional)\end{center}
Thank you!
\else
\ifisDipE
\begin{center}\bf Acknowledgements (optional)\end{center}
Thank you!
\else
\ifisDipD
\begin{center}\bf Danksagung (optional)\end{center}
Danke!
\else
\ifisBach
\begin{center}\bf Danksagung (optional)\end{center}
Danke!
\else
\begin{center}\bf Danksagung (optional)\end{center}
Danke!
\fi
\fi
\fi
\fi
\par\vfil\null
%
\newpage
%
\ifisDiss
\renewcommand{\contentsname}{Table of Contents}
\else
\ifisDipE
\renewcommand{\contentsname}{Table of Contents}
\else
\ifisDipD
\renewcommand{\contentsname}{Inhaltsverzeichnis}
\else
\ifisBach
\renewcommand{\contentsname}{Inhaltsverzeichnis}
\else
\renewcommand{\contentsname}{Inhaltsverzeichnis}
\fi
\fi
\fi
\fi
\tableofcontents
%
\ifisDiss
\chapter*{Abbreviations}
%\thispagestyle{empty}
\markboth{Abbreviations}{}
%
\begin{bfscript}{Somethings}
% enter your common and not so comman abbrev. here
\item[ABB] Abbreviation
\item[ABB] Abbreviation
\item[ABB] Abbreviation
\item[ABB] Abbreviation
\item[ABB] Abbreviation
\end{bfscript}
\else
\ifisDipE
\chapter*{Abbreviations}
%\thispagestyle{empty}
\markboth{Abbreviations}{}
%
\begin{bfscript}{Somethings}
% enter your common and not so comman abbrev. here
\item[ABB] Abbreviation
\item[ABB] Abbreviation
\item[ABB] Abbreviation
\item[ABB] Abbreviation
\item[ABB] Abbreviation
\end{bfscript}
\else
\ifisDipD
\chapter*{Abk�rzungen}
%\thispagestyle{empty}
\markboth{Abk�rzungen}{}
%
\begin{bfscript}{Irgendwas}
% enter your common and not so comman abbrev. here
\item[ABB] Also Bin Baff
\item[ABB] Also Bin Baff
\item[ABB] Also Bin Baff
\item[ABB] Also Bin Baff
\item[ABB] Also Bin Baff
\end{bfscript}
\else
\ifisBach
\chapter*{Abk�rzungen}
%\thispagestyle{empty}
\markboth{Abk�rzungen}{}
%
\begin{bfscript}{Irgendwas}
% enter your common and not so comman abbrev. here
\item[ABB] Also Bin Baff
\item[ABB] Also Bin Baff
\item[ABB] Also Bin Baff
\item[ABB] Also Bin Baff
\item[ABB] Also Bin Baff
\end{bfscript}
\else
\chapter*{Abk�rzungen}
%\thispagestyle{empty}
\markboth{Abk�rzungen}{}
%
\begin{bfscript}{Irgendwas}
% enter your common and not so comman abbrev. here
\item[ABB] Also Bin Baff
\item[ABB] Also Bin Baff
\item[ABB] Also Bin Baff
\item[ABB] Also Bin Baff
\item[ABB] Also Bin Baff
\end{bfscript}
\fi
\fi
\fi
\fi
%
\newpage
\pagenumbering{arabic}


% defines the pagestyle of the document to fancy package
\pagestyle{fancy}

% redefinitions of the plain style for the first pages of chapters.
\fancypagestyle{plain}{%
\fancyhf{}%
\fancyfoot[C]{\thepage}%
\renewcommand{\headrulewidth}{0pt}%
\renewcommand{\footrulewidth}{0pt}}

% twosided Head: {Chaptertitle                      }{                      Chaptertitle}
\renewcommand{\chaptermark}[1]{\markboth{}{\slshape\small{#1}}}
\renewcommand{\sectionmark}[1]{}

% After the preface use arabic numbers for pagecounting
% is already defined in titlepage.tex

% vertical separaiont of paragraphs
\setlength{\parskip}{1.5ex plus 1ex minus 0.5ex}

% include your chapters here.
% Do not use number words in the filenames, you may run into trouble reconfiguring your
% document otherwise!
%%%%%%%%%%%%%%%%%%%%%%%%%%%%%%%%%%%%%%%%%%%%%%%%%%%%%%%%%%%%%%%%%%%%%%%
%                                                                     %
%   Institut f�r Computertechnik, Gusshausstra�e 27-29, 1040 Wien     %
%                                                                     %
% Basisdokument f�r englischsprachige Dissertationen am ICT im pdf-   %
% Format                                                              %
%                                                                     %
%	Basiert auf der Dokumentklasse report.                        %
%                                                                     %
% Autor: Dietmar Bruckner                                             %
% Version:                                                            %
% Datum: 28. Aug. 2007                                                %
%                                                                     %
% --- Einf�hrung.tex ---                                              % 
%                                                                     %
%%%%%%%%%%%%%%%%%%%%%%%%%%%%%%%%%%%%%%%%%%%%%%%%%%%%%%%%%%%%%%%%%%%%%%%
% Sample Chapter for the use of the template                          %
%%%%%%%%%%%%%%%%%%%%%%%%%%%%%%%%%%%%%%%%%%%%%%%%%%%%%%%%%%%%%%%%%%%%%%%
\ifisDiss
\chapter{Introduction}
\label{cha:Introduction}

\paragraph{Dear candidate,}

you 
\else
\ifisDipE
\chapter{Introduction}
\label{cha:Introduction}

\paragraph{Dear candidate,}

you 
\else
\ifisDipD
\chapter{Einleitung}
\label{cha:Introduction}

\paragraph{Sehr geehrteR KandidatIn,}

ein Gro�teil des folgenden Textes ist in englischer Sprache verfasst, was jedoch nicht weiter st�ren sollte, da er sowieso gel�scht wird. In Folgenden soll Ihnen mit Hilfe von viel Beispielmaterial der Einstieg in die Welt von \LaTeX erleitert werden.


You 
\else
\ifisBach
\chapter{Einleitung}
\label{cha:Introduction}

\paragraph{Sehr geehrteR KandidatIn,}

ein Gro�teil des folgenden Textes ist in englischer Sprache verfasst, was jedoch nicht weiter st�ren sollte, da er sowieso gel�scht wird. In Folgenden soll Ihnen mit Hilfe von viel Beispielmaterial der Einstieg in die Welt von \LaTeX erleitert werden.


You 
\else
\chapter{Einleitung}
\label{cha:Introduction}

\paragraph{Sehr geehrteR KandidatIn,}

ein Gro�teil des folgenden Textes ist in englischer Sprache verfasst, was jedoch nicht weiter st�ren sollte, da er sowieso gel�scht wird. In Folgenden soll Ihnen mit Hilfe von viel Beispielmaterial der Einstieg in die Welt von \LaTeX erleitert werden.


You 
\fi
\fi
\fi
\fi
have taken your decision to wirte your master theses with the help
of \LaTeX. This template was desigend to support you during the
usage of this tool and to ensure that all theses of the institute \citeweblink{ICT07} may
look similar.

This example of a chapter demonstrates the
common textelements like headings, lists or captions of figures and tables.

Please keep in mind that the preface of your thesis has a normalized
format starting at the titlepage and the following four sections
abstract, Kurzfassung, acknowledgement. This format is the same for
the whole faculty and must not be changed except of your own personal
datas.

Important: To use two different literature lists - as required from now on at the ICT - you have to process both .bib files with bibtex. This "TeXnicCenter" project has the required command "path\_to\_bibtex/bibtex path\_to\_bib\_file/weblinks" defined as post processing command. If you use another IDE or makefiles be sure to take care of the second .bib file.

At the end of this introductional words I may wish you all the best
for your work.
\section{Section a}
Some text. Please replace by your own!.
Some text. Please replace by your own!
A Cite \cite{Mcpu96}.
\section{Section b}
Some text. Please replace by your own!.
Some text. Please replace by your own!
A Cite \cite{Mcpu96}.

lorem ipsum. lorem ipsum. lorem ipsum. lorem ipsum. lorem ipsum. Example for web links: (German) help for \TeX can be found in the Dante FAQs \citeweblink{dante}.

Some text. Please replace by your own!
Some text. Please replace by your own!
Some text. Please replace by your own!
Some text. Please replace by your own!
Some text. Please replace by your own!
Some text. Please replace by your own!
Some text. Please replace by your own!
Some text. Please replace by your own!
Some text. Please replace by your own!
Some text. Please replace by your own!
Some text. Please replace by your own!
Some text. Please replace by your own!
Some text. Please replace by your own!
Some text. Please replace by your own!
Some text. Please replace by your own!
Some text. Please replace by your own!
Some text. Please replace by your own!
Some text. Please replace by your own!
Some text. Please replace by your own!
Some text. Please replace by your own!
Some text. Please replace by your own!
Some text. Please replace by your own!
Some text. Please replace by your own!
Some text. Please replace by your own!


\subsection{sdv}
Some text. Please replace by your own!
Some text. Please replace by your own!
Some text. Please replace by your own!
Some text. Please replace by your own!
Some text. Please replace by your own!
Some text. Please replace by your own!
\subsection{sddsaxcv}
Some text. Please replace by your own!
Some text. Please replace by your own!
Some text. Please replace by your own!
Some text. Please replace by your own!
Some text. Please replace by your own!
Some text. Please replace by your own!
Some text. Please replace by your own!
Some text. Please replace by your own!
Some text. Please replace by your own!
Some text. Please replace by your own!
Some text. Please replace by your own!
Some text. Please replace by your own!
Some text. Please replace by your own!
Some text. Please replace by your own!

\begin{itemize}
\setlength{\itemsep}{0.0mm}
\item 1 Some text. Please replace by your own!
\item 2 Some text. Please replace by your own!
\item 3 Some text. Please replace by your own! 
\end{itemize}

\renewcommand{\labelenumi}{\alph{enumi})}
\begin{enumerate}
\item Some text. Please replace by your own! 
\item Some text. Please replace by your own!
\end{enumerate}
\renewcommand{\labelenumi}{\arabic{enumi}.}

\begin{enumerate}
\item Some text. Please replace by your own! 
\item Some text. Please replace by your own!
\end{enumerate}

Some text. Please replace by your own!
Some text. Please replace by your own!
Some text. Please replace by your own!
Some text. Please replace by your own!
Some text. Please replace by your own!
Some text. Please replace by your own!
Some text. Please replace by your own!
Some text. Please replace by your own!
Some text. Please replace by your own!
Some text. Please replace by your own!
Some text. Please replace by your own!
Some text. Please replace by your own!
Some text. Please replace by your own!
Some text. Please replace by your own!
Some text. Please replace by your own!
Some text. Please replace by your own!
Some text. Please replace by your own!

\paragraph{Paragraph}
Some text. Please replace by your own!
Some text. Please replace by your own!
Some text. Please replace by your own!
Some text. Please replace by your own!
Some text. Please replace by your own!
Some text. Please replace by your own!
Some text. Please replace by your own!
Some text. Please replace by your own!
Some text. Please replace by your own!
Some text. Please replace by your own!


A link to a picture far behind\footnote{Figure \ref{2dpdfgeneral}.}

Some text. Please replace by your own!
Some text. Please replace by your own!
Some text. Please replace by your own!
Some text. Please replace by your own!
Some text. Please replace by your own!
Some text. Please replace by your own!
Some text. Please replace by your own!
Some text. Please replace by your own!
Some text. Please replace by your own!
Some text. Please replace by your own!
Some text. Please replace by your own!
Some text. Please replace by your own!

\paragraph{Paragraph}
Some text. Please replace by your own!
Some text. Please replace by your own!
Some text. Please replace by your own!
Some text. Please replace by your own!
Some text. Please replace by your own!
Some text. Please replace by your own!
Some text. Please replace by your own!
Some text. Please replace by your own!
Some text. Please replace by your own!
Some text. Please replace by your own!
Some text. Please replace by your own!
Some text. Please replace by your own!
Some text. Please replace by your own!
Some text. Please replace by your own!
Some text. Please replace by your own!
Some text. Please replace by your own!

\chapter{Sample Text: Probability Density Estimation}
\label{pde}
For finding similarities within sensor data, statistics provides us with a number of methods. For the scope of this thesis, I have choosen to use methods from pattern recognition, probability density estimation and machine learning. Among other things, these fields provide methods for describing data from a data source and adapting a (probabilistic) model's structure suitable for modeling different types and combinations thereof. This chapter gives an overview of the necessary terms and algorithms for the above mentioned topics.

To give an introduction in statistical methods in building automation, I will discuss the example of a temperature sensor located in an office. The sensor is attached to a sensor node (mote) that has a programmable hysteresis, in our case we use $\pm 0.1 $�C. If the sensor recognizes a temperature difference greater than that, it launches a message that is transported via the network to a data sink mote. If the temperature stays stable, the mote sends ``keep alive'' messages with the current temperature every 15 minutes.
This is the only possibility to find out if the sensor is still in operation. 
\begin{figure}
\setlength{\unitlength}{1mm}
\begin{center}
\includegraphics[width=15cm, height=10cm]{temperature}
\end{center}

\caption{Data from a temperature sensor collected over one day. In periods where the temperature is changing quickly the sensor delivers more values than otherwise because of its hysteresis. In periods of constant temperature the sensor gives ``keep alive's'' every 15 minutes.\label{temp-fig}}
\end{figure}
The sink mote is connected to a PC via USB. On the PC the message format is converted and the data stored in a data base.\\ The reason for this comprehensive description of the data flow lies in the fact that the way the data is generated, transported and stored influences the choice of the statistical model. The above-mentioned sensor uses a combination of synchronous and asynchronous messages. An example of the sensor data for one day in the mentioned office is given in \Fig{temp-fig}.

Probability density estimation is commonly seen as the problem of modeling a probability density function (pdf) $p(\vec{x})$, given a finite ($N$) number of data points\footnote{In the field of statistics a data point is a single typed measurement. The term typed refers to the data type of the data point regardless of the type of the data source. In the above-stated example the type would be a three-dimensional vector of real values.} $\vec{x}^{n}$, $n=1, ..., N$. $\vec{x}$ being a $d$ dimensional data point (e.g. brightness, humidity and temperature in case of a combined senor). The following subsections concerning pdfs follow the notation of \citep{Bis95}.\\
I will explain the notation here in symbiosis with the introduction of Bayes' theorem (which is needed in \sect{param}).
Bayes' theorem
\begin{equation}P(A=a|B=b) = \frac{P(B=b|A=a)P(A=a)}{P(B=b)}\label{bayestheorem}\end{equation}
relates the conditional and marginal probabilities of two stochastic events. An event in this case happened when a random variable (e.g. $A$) takes a particular value (e.g. $a$), written $A=a$ or $B=b$. $P(B=b|A=a)$ is the conditional probability of the random variable $B$ being $b$ given that the random variable $A$ has taken a fixed value of $a$.
The common used names for above terms are
\begin{itemize}
\item{$P(A=a)$ is called marginal probability of $A$ being $a$, or prior probability because it does not see about the influence of $B$.}
\item{$P(A=a|B=b)$ is the conditional probability of $A$ being $a$ given that $B$ definitely took $b$. In the Bayesian vocabulary it is also called posterior probability because it takes the impact of the random variable $B's$ value $b$ on $A$ taking its value $a$ into account.}
\item{$P(B=b|A=a)$ is the conditional probability of $B=b$ given $A=a$.}
\item{$P(B=b)$ is the prior or marginal probability of $B=b$ and acts here as a normalizing constant.}
\end{itemize}
If the random variable can take several events $\{A=a_i\}$, the posterior for a single event can be obtained by
\begin{equation}P(A=a_i|B=b) = \frac{P(B=b|A=a_i)P(A=a_i)} {\sum_j{P(B=b|A=a_j)P(A=a_j)}}.\label{bayestheorem2}\end{equation}
Uppercase letters (e.g. $P(A=a)$) are used for probabilities of discrete events, whereby lowercase letters (e.g. $p(x)$) are used for probability density functions. Multidimensional values are stated as vector (eg. $p(\vec{x})$).\par
In the theory of probability density estimation, there are three approaches to density estimation: parametric, non-parametric and semi-parametric. The former supposes a particular density function and estimates its parameters for the observed data. Unfortunately, there is no guarantee that the assumption regarding the form of the chosen function models the actual data well. By contrast, non-parametric density estimation makes no assumption at all and therefore lets \emph{the data speak on its own}. The drawback is that the automatic determination of the form leads to large numbers of parameters in the result, typically growing with the size of the data set. Mixture models on the other hand are one particular form of semi-parametric models. Compared to parametric and non-parametric models, semi-parametric models are not restricted to specific functional forms and the size of the model only grows with the complexity of the problem to be solved, not with the size of the data set. The only disadvantage is that the training process of the model is computationally more intensive.
\section{Parametric Methods}
\label{param}
The parametric approach assumes that the probability density $p(\vec{x})$ can be expressed in terms of a specific functional form which contains a number of adjustable parameters. These parameters can then be optimized to find the best fit of the proposed pdf to the actual data. The most common parametric model is the Gaussian or normal distribution. The method of parametric probability density estimation can be described with any function. Fortunately, the normal distribution has several convenient properties. Therefore, I will discuss it here in detail.\\
The well known density function of the normal distribution for a single variable is given by 
\begin{equation}p(x)=\frac{1}{\sqrt{2\pi\sigma^2}}\exp\left\lbrace -\frac{(x-\mu)^2}{2\sigma^2}\right\rbrace. \label{gauss1}\end{equation}
The parameters $\mu$ and $\sigma^2$ are called \emph{mean} and \emph{variance}, respectively. The factor in front of the exponent ensures the summation to $1$ when integrating over $\mathds{R}$. The mean and variance of the normal distribution are equivalent to its expected value and its $2^{nd}$ moment
\begin{displaymath}\mu=E \left[ x \right] =\int_{-\infty}^\infty x p(x) dx\end{displaymath}
\begin{displaymath}\sigma^2=E \left[ (x-\mu)^2\right] =\int_{-\infty}^\infty (x-\mu)^2 p(x) dx\end{displaymath}
where $E[\cdot]$ denotes the expectation. In case of multidimensional data the $d$-dimensional density function is given by
\begin{displaymath}p(\vec{x}) = \frac{1}{\sqrt{(2\pi)^d|\boldsymbol{\Sigma}|}}\exp{\left\{-\frac{1}{2}(\vec{x}-\boldsymbol{\mu} )^T \boldsymbol{\Sigma}^{-1} (\vec{x}-\boldsymbol{\mu}) \right\}}
\label{multigauss}\end{displaymath}
where $\boldsymbol{\mu}$ and $\boldsymbol{\Sigma}$ are a $d$-dimensional vector and a $d\times d$ \emph{covariance matrix}, respectively. $|\boldsymbol{\Sigma}|$ is the determinant of $\boldsymbol{\Sigma}$ and the factor in front of the exponent again ensures summation to unity. Mean and variance also satisfy the expectations 
\begin{displaymath}\boldsymbol{\mu}=E \left[ \vec{x} \right]\end{displaymath}
\begin{equation}\boldsymbol{\Sigma}=E \left[ (\vec{x}-\boldsymbol{\mu}) (\vec{x}-\boldsymbol{\mu})^T\right].\label{sigmama}\end{equation}
\section{a second Parametric Methods section}
this one is to remind you that one single headline in a particular depth is forbidden! always use at least two.
\begin{figure}
\setlength{\unitlength}{1mm}
\begin{center}
\begin{picture}(70,60)
\put(0,5){\vector(1,0){70}}
\put(65,0){$x_1$}
\put(5,0){\vector(0,1){60}}
\put(0,55){$x_2$}
\put(-25,-20){\rotatebox{30}{\makebox(60,40){
\thicklines
%\put(30,20){\ellipse{50}{30}}
\qbezier(55.0, 20.0)(55.0, 26.2132)(47.6777, 30.6066)
\qbezier(47.6777, 30.6066)(40.3553, 35.0)(30.0, 35.0)
\qbezier(30.0, 35.0)(19.6447, 35.0)(12.3223, 30.6066)
\qbezier(12.3223, 30.6066)(5.0, 26.2132)(5.0, 20.0)
\qbezier(5.0, 20.0)(5.0, 13.7868)(12.3223, 9.3934)
\qbezier(12.3223, 9.3934)(19.6447, 5.0)(30.0, 5.0)
\qbezier(30.0, 5.0)(40.3553, 5.0)(47.6777, 9.3934)
\qbezier(47.6777, 9.3934)(55.0, 13.7868)(55.0, 20.0)
\multiput(30,20)(1,0){30}{\line(1,0){0.5}}
\multiput(30,20)(0,1){20}{\line(0,1){0.5}}
\put(59,20){\vector(1,0){1}}
\put(54,20){\vector(1,0){1}}
\put(31,20){\vector(-1,0){1}}
\put(30,39){\vector(0,1){1}}
\put(30,34){\vector(0,1){1}}
\put(30,21){\vector(0,-1){1}}
}}}
\put(-35,-10){\makebox(70,60){
\put(62,44){$\vec{u}_1$}
\put(28,49){$\vec{u}_2$}
\put(45,40){$\lambda_1$}
\put(34,39){$\lambda_2$}}}
\end{picture}
\end{center}
\caption{General curve with constant probability density of a $2D$ Gaussian pdf. The ellipse is aligned according to the eigenvectors $\vec{u}_i$ of the covariance matrix $\boldsymbol{\Sigma}$. The length of the axis is proportional to the corresponding eigenvalues $\lambda_i$.}
\label{2dpdfgeneral}
\end{figure}
\chapter{Merkblatt f�r den Aufbau wissenschaftlicher Arbeiten}
This is the German version of G�schka's guidlines for writing scientific papers. It is put here as another sample text with several hints how to use \LaTeX. The English verison can be found in the next chapter.
\section*{Zusammenfassung}
Wissenschaftliche Arbeiten mit anspruchsvollem Inhalt sollten auch in ihrer formalen Struktur bestimmten Richtlinien entsprechen. Das garantiert, dass der Inhalt auch effizient vermittelt wird. Dabei kommt vor allem der Kurzfassung als meistgelesenem Teil der Arbeit besondere Bedeutung zu: Sie muss die Essenz der Arbeit vorwegnehmen und zugleich zum Lesen verlocken. Weitere wichtige Elemente der Arbeit sind die Einleitung, der Hauptteil und schlie�lich der Schluss mit Zusammenfassung und Ausblick. Dabei erleichtern vor allem Beispiele und graphische Darstellungen das Verst�ndnis. 
In den Anhang geh�ren allf�llige Verzeichnisse, Listings oder ein Index sowie erg�nzende Informationen, die vom roten Faden der Arbeit abweichen. Niemals fehlen darf das Literaturverzeichnis, auf das man zumindest im Abschnitt �ber die verwandten Arbeiten rege verweisen sollte. Wenn man diese einfachen Regeln beachtet, kann man sich getrost auf den Inhalt konzentrieren: Denn auf den Inhalt 
kommt es an!
\section*{Danksagung}
Mein Dank gilt meinen Kollegen Thomas Kittenberger, Richard Schmalek und Klemens Urban f�r das Beisteuern zus�tzlicher Informationen und Anregungen sowie f�r die Bereitschaft zur Diskussion.
\section*{Inhaltsverzeichnis}
siehe Inhaltsverzeichnis am Beginn des Dokuments.
Originaltext:\\
Das Inhaltsverzeichnis ist hier nur zur Illustration angef�hrt. �blicherweise wird es bei kurzen Artikeln weggelassen.
\section{Die grundlegende Struktur}
Das Wichtigste an einer wissenschaftlichen Arbeit ist ihr Inhalt. Auch der sch�nste formale Aufbau kann nicht �ber einen schwachen Inhalt hinwegt�uschen: Sp�testens, wenn die erste Blendwirkung vergangen ist und der Leser sich intensiver mit
der Arbeit auseinandersetzt, schl�gt die Stunde der Wahrheit. Allerdings wird umgekehrt eine hervorragende Arbeit mit umst�ndlichem oder un�bersichtlichem Aufbau gar nicht so weit kommen, gelesen zu werden: Denn der erste Eindruck einer Arbeit auf den potentiellen Leser wird durch den formalen Aufbau vermittelt, und erst aufgrund dieses Eindrucks entscheidet der Leser, ob er sich n�her mit einer Arbeit besch�ftigen m�chte. \\
Gerade in einer Zeit, da zu jedem Fachgebiet eine Unmenge von Arbeiten zur Verf�gung steht, muss man den Aufwand, den man in den formalen Aufbau einer Arbeit investiert, als jene Anstrengung ansehen, die notwendig ist, um beim Leser �berhaupt in die engere Wahl zu kommen. Daher ist als erstes ein dem Leser vertrauter, weil allgemein �blicher, Aufbau Voraussetzung: Der Leser muss rasch erkennen k�nnen, ob der Artikel f�r ihn �berhaupt interessant ist, und wenn ja, wo er die f�r ihn interessanten Teile im Text findet. Wie sieht nun dieser \textit{standardisierte Aufbau} aus?
\begin{itemize}
	\item Titel (Deckblatt)
	\item Kurzfassung
	\item Danksagung
	\item Inhaltsverzeichnis
	\item Einleitung
	\item Hauptteil
	\item Schluss
	\item Anhang
\end{itemize}
Damit ist der Aufbau einer wissenschaftlichen Arbeit grob umrissen, in den folgenden Abschnitten werden die wichtigsten Gliederungselemente noch genauer besprochen. Dieser Artikel ist �brigens selbst ein Beispiel f�r den formalen Aufbau einer Arbeit, allerdings unter Ber�cksichtigung der Besonderheiten f�r kurze Artikel, siehe Abschnitt \ref{ka} auf Seite \pageref{ka}.
\section{Die Komponenten der Arbeit}
\subsection{Der Titel}
Der Titel ist die Kurzfassung der Kurzfassung! Er soll dem Leser in aller K�rze sagen, was er erwarten kann. Blumige Phantasietitel oder Wortspiele sind zwar lustig, bieten aber meist keine Entscheidungshilfe, ob man sich �berhaupt die Kurzfassung des Artikels besorgen soll oder nicht. Vorsicht ist auch bei Eigennamen oder selbstdefinierten Begriffen geboten. Gegebenenfalls kann ein Untertitel hilfreich sein, um mehr �ber den Inhalt zu vermitteln, ohne den eigentlichen Titel zu lang werden zu lassen. \\
Bei l�ngeren Arbeiten erh�lt der Titel ein eigenes Deckblatt. Dieses kann je nach Art der Arbeit sehr unterschiedlich aussehen und ist oft auch vorgegeben. Zumeist enth�lt es neben dem Namen des Autors auch seine Dienstadresse, evtl. mit Logo, das Datum sowie die m�gliche Erreichbarkeit via Telefon, Fax oder Email. Bei Arbeiten von Studenten sollten au�er dem Namen die Matrikelnummer, die Bezeichnung und das Semester der Lehrveranstaltung sowie das Abgabedatum nicht fehlen.
\subsection{Die Kurzfassung}
Manche behaupten, die Kurzfassung sei �berhaupt der wichtigste Teil einer wissenschaftlichen Arbeit.Das mag zwar etwas �bertrieben sein; zweifellos ist sie aber der \textit{meistgelesene} Teil einer Arbeit. Ihre Aufgaben sind zweierlei: 
\begin{enumerate}
	\item Sie soll dem aufgrund des Titels interessierten Leser mehr Information geben. Damit soll die Entscheidung erleichtert werden, ob der Artikel f�r den Leser interessant ist oder nicht. Ein bi�chen Werbung in eigener Sache kann dabei nicht schaden; man kann dem Leser ruhig etwas Gusto auf den Artikel machen. Sinnlos ist aber, eine �bertriebene Erwartungshaltung zu wecken, die vom Artikel nicht befriedigt wird. 	
\item F�r den am Thema weniger interessierten Leser soll die Kurzfassung aber gerade noch soviel Information enthalten, dass er das Wesentliche erf�hrt, ohne den Artikel selbst lesen zu m�ssen.
\end{enumerate}
Damit die Kurzfassung ihre Aufgaben erf�llen kann, muss sie zumindest folgende Punkte beinhalten:
\begin{itemize}
	\item Den Themenkreis und die behandelte Problematik, um die \textit{Motivation} der Arbeit zu erkl�ren.
	\item Den L�sungsansatz und die Methodik der Arbeit.
	\item Die Essenz der L�sung, also die wichtigsten Ergebnisse und Erkenntnisse.
\end{itemize}
Damit wird die Kurzfassung zu einem eigenst�ndigen Kurzartikel zum selben Thema wie die Arbeit. Dabei soll die Kurzfassung jedoch einen Umfang von ca. 200 W�rtern nicht �berschreiten, bei langen Berichten oder Diplomarbeiten ist maximal eine ganze Seite zul�ssig. \\
Die Kurzfassung ist keine Zusammenfassung und sollte auch nicht so bezeichnet werden. Der wesentliche Unterschied zwischen Kurzfassung und Zusammenfassung liegt darin, dass man bei der Zusammenfassung den Inhalt\footnote{Vor allem die Begriffswelt und die dem Themenbereich eigenen Methoden.} der Arbeit voraussetzen darf, da sie ja erst am Ende der Arbeit steht. Die Kurzfassung hingegen steht am Beginn der Arbeit und der Inhalt ist dem Leser noch unbekannt. \\
Schlimmer ist allerdings, wenn die Kurzfassung zur Inhaltsangabe oder Gliederungsbeschreibung degeneriert: Oft wird dann ausgehend vom Ansatz aufgez�hlt, was alles in der Arbeit behandelt wird, man erf�hrt aber nicht, was dabei herausgekommen ist. Dieser Fehler entsteht unter anderem dadurch, dass die Kurzfassung als erstes geschrieben wird, wenn der Autor selbst oft den Inhalt noch nicht exakt festgelegt hat. Es ist zwar g�nstig, die Kurzfassung zu Beginn zu schreiben, man darf aber nicht darauf vergessen, die Kurzfassung zuletzt nochmals zu �berarbeiten, aber dennoch nicht mit einer Zusammenfassung zu verwechseln. Auf eine Inhaltsangabe oder Gliederungs�bersicht\footnote{Nicht zu verwechseln mit dem Inhaltsverzeichnis! In der \textit{Inhaltsangabe} oder \textit{Gliederungs�bersicht} werden der inhaltliche Aufbau und die formale Struktur der nachfolgenden Kapitel beschrieben.} muss man dennoch nicht verzichten: Diese erfolgt �blicherweise am Ende der Einleitung oder �berhaupt am Ende jedes Kapitels f�r die folgenden. \\
Um die Einordnung ver�ffentlichter Arbeiten in Bibliotheken oder Datenbanken zu erleichtern, ist es sinnvoll, eine Reihe von Schl�sselbegriffen anzugeben, nach denen die Arbeit sp�ter aufgefunden werden kann. Wenn man diese Begriffe nicht selbst angibt, werden sie sp�ter von irgendjemandem der Kurzfassung entnommen. Da aber niemand besser als der Verfasser wei�, welche Begriffe f�r die Arbeit am treffendsten sind, sollte man diese Begriffe stets selbst angeben. 
\subsection{Die Danksagung}
Diese soll weder peinlich wirken noch zur Pflicht�bung werden -- sie ist nicht zwingend notwendig! Wenn man aber den Wunsch hat, verschiedenen Personen den Dank f�r ihre Unterst�tzung auszusprechen, so sollte man auch die Art der Unterst�tzung nennen. Auf jeden Fall sind all jene in der Danksagung zu erw�hnen, die zur Arbeit beigetragen haben, nicht aber die Autoren, falls es mehrere sind. \\
Ob jemand nun in der Danksagung oder als Autor aufscheint, kann zum Gegenstand langer Diskussionen werden. In jedem Fall sollte aber ein Autor �ber den \textit{gesamten} Inhalt der Arbeit genau Bescheid wissen.
\subsection{Das Inhaltsverzeichnis}
Das Inhaltsverzeichnis ist ein Spiegel der Gliederung der Arbeit und wird fast genauso oft gelesen wie die Kurzfassung. Es l�sst das Schwergewicht der Arbeit erkennen und auch die Methodik, wie an das Thema herangegangen wurde. Daher sollten die einzelnen Kapitel und Abschnitte m�glichst aussagekr�ftig betitelt werden, und nicht einfach nur ``Einleitung" oder ``Schluss". Da alle g�ngigen Textverarbeitungen eine Generierung des Inhaltsverzeichnisses erm�glichen, sollte es bez�glich der Seitennumerierung keine Inkonsistenzen geben. Bei sehr kurzen Artikeln (unter 10 Seiten) \textit{kann} das Inhaltsverzeichnis auch entfallen. \\
An dieser Stelle noch ein Wort zur Gliederungstiefe und Gliederungshierarchie: Man findet manchmal �berschriften der Form ``\textit{2.3.A.IV.8.2.b Induktiver Beweis}" mit sehr tiefen Strukturen. Diese sollte man eher vermeiden, da sie nicht zur �bersichtlichkeit beitragen. Die Gliederungszahl sollte maximal dreibis vierstellig sein, darunter kann man noch maximal ein bis zwei unnumerierte Hierarchieebenen verwenden. Sollte sich die Notwendigkeit nach tieferen Strukturen ergeben, ist eine grunds�tzliche Umstrukturierung der Arbeit zu erw�gen. Erw�hnenswert ist noch, dass die Verzeichnisse (Inhaltsverzeichnis, Stichwortverzeichnis, Literaturverzeichnis, etc \dots) keine Gliederungsnummern erhalten.
\subsection{Die Einleitung}
\label{de}
Mit der Einleitung beginnt der eigentliche Inhalt der Arbeit. Man beginnt g�nstigerweise damit, den \textit{Themenkreis} der Arbeit grob darzustellen, um den Leser mit dem Umfeld der Arbeit vertraut zu machen. Dabei ist es nicht notwendig, bei Adam und Eva anzufangen. Vielmehr soll sich der Einstieg an dem bei der Zielgruppe vorauszusetzenden Wissen orientieren. Bei wissenschaftlichen Arbeiten muss man vom Leser annehmen d�rfen, dass er selbst einschl�gig vorgebildet ist. Es ist also zum Beispiel nicht notwendig, Begriffe zu erl�utern, die in der jeweiligen Fachrichtung zum Allgemeingut geh�ren, au�er man ist dabei, eine Grundlagenabhandlung oder ein Vorlesungsskriptum zu schreiben. \\
Sehr wohl interessant ist aber, welchem (industriellen) Projekt die Arbeit zuzuordnen ist und welche Rolle sie darin spielt. Hier muss unter anderem hervorgehoben werden, warum das Themengebiet im allgemeinen oder die Arbeit im speziellen es �berhaupt wert sind, dass man sich damit auseinandersetzt. Die \textit{Motivation} f�r die Arbeit selbst muss daraus klar erkennbar sein. \\
Hat man gekl�rt, welchem Themenkreis und Projekt die Arbeit zuzuordnen ist, muss man erl�utern, wie die Arbeit in diesen Themenkreis eingebettet ist und was \textit{Stand der Technik} ist. Aus den allgemeinen Problemstellungen des Themenkreises ist zunchst die \textit{Aufgabenstellung} der Arbeit abzuleiten und im Detail anzuf�hren. Dieser Punkt ist sehr wichtig, denn hier beginnt der \textit{rote Faden}, der sich durch die gesamte Arbeit ziehen sollte. Damit ist gemeint, dass die Arbeit zielstrebig verl�uft, und sich nicht in Nebens�chlichkeiten verliert: Der Leser muss zu jedem Zeitpunkt erkennen k�nnen, wie ein bestimmter Teil der Arbeit mit der anf�nglichen Aufgabenstellung zusammenh�ngt. Zu diesem Zweck kann man in den einzelnen Kapiteln auf bestimmte Punkte der exakten Definition der Aufgabenstellung verweisen. \\
Die Definition der Aufgabenstellung findet ihren Gegenpol dann in der Zusammenfassung, wo der rote Faden mit einem Vergleich endet, ob und wie die anf�nglichen Aufgabenstellungen von der Arbeit nun tats�chlich erf�llt wurden. \\
Man kann die Einleitung auch dazu ben�tzen, wichtige \textit{Begriffe} und \textit{Abk�rzungen} zu definieren und zu erl�utern, oder ganz allgemein, die f�r diesen Themenkreis speziellen Konzepte und \textit{Methoden} vorzustellen. Dabei ist besonders auf eine klare Abgrenzung deutscher und fremdsprachiger Begriffe zu achten. N�heres dazu im Anschnitt \ref{sr} auf Seite \pageref{sr}. \\
Den Schluss der Einleitung kann ein �berblick �ber den Inhalt der folgenden Kapitel bilden. Dabei sollte �ber die Titel der folgenden Kapitel hinausgehende Information vermittelt werden, etwa eine bestimmte Systematik der Kapiteluntergliederung. Liegt das Schwergewicht der Arbeit im didaktischen Bereich, also etwa bei Skripten oder Lehrb�chern, kann man auch jedes folgende Kapitel mit einem solchen �berblick enden lassen. Dabei sollte man aber nicht �bertreiben, weil man sonst dazu neigt, �berheblich zu wirken.
\subsection{Der Hauptteil}
Der Hauptteil einer umfangreicheren Arbeit unterliegt folgender Gliederungshierarchie.
\begin{itemize}
	\item Teile
	\item Kapitel
	\item Abschnitte
	\item Unterabschnitte
	\item Unter-Unterabschnitte
	\item Paragraphen
	\item Unterparagraphen
\end{itemize}
Die Arbeit besteht also aus mehreren Kapiteln, eventuell sind die Kapitel zu Teilen zusammengefasst. Jedes Kapitel besteht aus Abschnitten, die ihrerseits weiter unterteilt sein k�nnen. Wenn die Arbeit k�rzer ist, wird man eine Kapitelgliederung vermeiden und direkt in Abschnitte unterteilen, so wie zum Beispiel bei \textit{diesem} Artikel selbst. Insgesamt werden sich je nach Art der Arbeit etwa folgende Elemente im Hauptteil befinden:
\begin{description}
	\item[Problemstellung:] Es ist Geschmacksache, ob man die Problemstellung noch in der Einleitung oder als ersten Punkt des Hauptteils behandelt. Sie sollte jedenfalls immer vorhanden sein.
	\item[L�sungsansatz:] Dieser enth�lt die grundlegenden neuen Ideen, Methoden, Konzepte und Vorgangsweisen, die zur L�sung gef�hrt haben.
  \item[Theorie:] Der theoretische Teil enth�lt zum Beispiel Algorithmen und Datenstrukturen, Schaltbilder, mathematische  Herleitungen und Beweise, Syntaxbeschreibungen oder �hnliche Punkte.
  \item[Praxis:] Im praktischen Teil finden sich die Beschreibung von Realisierungen, wie Implementierungen, Schaltungsprototypen oder Anwendungen, um nur einige M�glichkeiten als Beispiele zu nennen.
	\item[Ergebnis:] Dieser Punkt enth�lt die Erkenntnisse, Ergebnisse und L�sungen der Arbeit, sofern vorhanden. Diese m�ssen auch in Hinblick auf die Aufgabenstellung \textit{bewertet} werden. Auch m�gliche Verbesserungen k�nnen hier bereits genannt werden, ebenso wie weiterf�hrende Arbeiten.
\end{description}
Der Unterschied zwischen Kapitel und Abschnitt besteht dabei weniger im Umfang, sondern mehr in der Art der weiteren Unterteilung: Im Gegensatz zu den Abschnitten sollten die einzelnen Kapitel ihrerseits einen einheitlichen Aufbau aufweisen, wobei hier die Bedingungen nicht so streng sind wie beim Gesamtdokument und au�erdem stark vom Inhalt der Arbeit abh�ngen. Zumindest sollte jedes Kapitel eine kurze Einleitung besitzen, in der dem Leser mitgeteilt wird, was er vom Kapitel erwarten kann, au�er man hat diese Information in einem �berblick am Ende des vorangegangenen Artikels untergebracht. Danach folgt der Inhalt des Kapitels, der vor allem durch Tabellen, Beispiele oder graphische Darstellungen so aufgelockert werden soll, dass dem Leser das Verst�ndnis erleichtert wird.\\
Oft ist es schwierig, die komplexe Informationsstruktur auf die notwendigerweise lineare Struktur der schriftlichen Arbeit abzubilden. Dabei kann man sich mit \textit{Querverweisen} behelfen, wenn man sich Redundanz ersparen will. Die Verwendung von Querverweisen hilft dem Leser beim Verst�ndnis der komplexeren Zusammenh�nge der Arbeit, sollte aber nicht �bertrieben werden, um den Leser nicht endg�ltig zu verwirren. \\
Verwendet man die Arbeiten anderer im eigenen Text, dann bedient man sich des \textit{Zitierens}: Entweder man zitiert \textit{inhaltlich} oder \textit{w�rtlich}, wobei zweiteres eine besondere Kennzeichnung erfordert, zum Beispiel Anf�hrungsstriche. Dabei wird man vor allem kurze Literaturstellen w�rtlich zitieren, w�hrend man l�ngere Literaturstellen eher sinngem�� zusammenfasst. In jedem Fall f�gt man aber einen Verweis auf das Literaturverzeichnis ein, z.B. \cite{Ert93}. Autorennamen im Text werden in Gro�buchstaben oder in einer ``Small Caps"-Schrift angef�hrt, etwa \textsc{Anton Ertl}. Ob man als Literaturverweis nur ein K�rzel oder die Autorennamen verwendet, ist Geschmackssache: Jedenfalls ehrt man die Autoren durch explizite Nennung des vollen Namens besonders. \\
Beim \textit{sekund�ren} Zitieren, wenn man also ein Zitat aus einer anderen Arbeit �bernimmt, muss man auch jene Arbeit nennen, aus der das Zitat selbst entnommen wurde, und nicht nur die Originalarbeit. Grunds�tzlich ist vom sekund�ren Zitieren aber abzuraten, man sollte in jedem Fall versuchen, die Originalliteratur zu erhalten und daraus dann direkt zitieren. \\
Durch Zitieren bietet man dem interessierten Leser die M�glichkeit, die Quelle sowie weiterf�hrende Literatur nachzulesen. Andererseits kann man sich mit einem Verweis auf Grundlagenliteratur die Abhandlung elementarer Sachverhalte ersparen, wenn diese f�r die Leser-Zielgruppe der Arbeit mit Masse bekannt sind (vergleiche dazu die Bemerkungen zur Zielgruppe in Abschnitt \ref{de} auf Seite \pageref{de}). Es ist �brigens eine sehr effiziente Methode der Literatursuche, sich zu einem Thema ein m�glichst aktuelles Werk zu suchen und dessen Literaturstellen durchzugehen. \\
Jedes Kapitel sollte mit einer kurzen Zusammenfassung enden, in der die wichtigsten Aussagen des Kapitels zusammengefasst und miteinander in Beziehung gebracht werden. Abschlie�end kann ein �berblick �ber die kommenden Kapitel folgen. 
\subsection{Der Schluss}
Der Schluss ist das letzte Kapitel der Arbeit vor dem Anhang und den Verzeichnissen. Er enth�lt zumindest die Zusammenfassung mit einer Reflexion und Bewertung der Aufgabenstellung, meist auch einen Ausblick oder eine Aufz�hlung verwandter Arbeiten. \\
Bei der \textit{Zusammenfassung} endet nun der bei der Einleitung begonnene rote Faden in einem Vergleich, ob und wie die anf�nglichen Aufgabenstellungen von der Arbeit nun tats�chlich erf�llt wurden. Dabei werden die wichtigsten Aussagen der gesamten Arbeit noch einmal aufgez�hlt, miteinander in Beziehung gebracht und \textit{bewertet}. Es ist keine Schande auch anzugeben, welche Punkte der Aufgabenstellung nicht oder nur unzureichend behandelt werden konnten, solange man schl�ssig nachweisen kann, dass es sich um sehr komplexe Aufgabenstellungen handelt, die den Umfang der Arbeit gesprengt h�tten\footnote{Wenn man zum Beispiel beweisen kann, dass ein Problem NP-vollst�ndig ist, also nondeterministisch polynomial l�sbar, dann ist es keine Schande, nur einen Algorithmus von exponentiellem Aufwand gefunden zu haben. Nur beweisen muss man das eben!}. �bertriebene Bescheidenheit wie ``\dots in meiner Arbeit wird eigentlich nichts wirklich gekl�rt \dots" ist dabei genauso fehl am Platz wie �bergro�es Eigenlob ``\dots meine Arbeit l�st alle Probleme ganz leicht mit einem Schlag \dots". Als Zweitfunktion dient die Zusammenfassung auch als ``Auffangbeh�lter" f�r jene Leser, die in der Mitte die Geduld verloren und den Rest des Textes �berbl�ttert haben: In der Zusammenfassung kann man solche Leser noch einmal von der Relevanz der Arbeit �berzeugen und sie eventuell sogar dazu �berreden, den Rest der Arbeit doch noch zu lesen. \\
Schlie�lich ist es sinnvoll anzugeben, wie sich nun die L�sung der Arbeit im Gesamtprojekt einf�gt. Im \textit{Ausblick} kann man auch noch erw�hnen, welche weiteren Schritte als n�chstes zu tun sind oder welche sonstigen Anwendungen f�r die Arbeit denkbar w�ren. Man formuliert damit die Aufgabenstellungen weiterf�hrender Arbeiten und kann auch schon L�sungsans�tze mit auf den Weg geben. \\
Den Abschnitt �ber die \textit{verwandten Arbeiten} kann man entweder am Ende der Einleitung unterbringen oder als eigenen Punkt an die Einleitung anschlie�en lassen. Dies ist vor allem dann sinnvoll, wenn man konkurrierende Arbeiten beschreibt. Dabei sollte man nicht vergessen zu erw�hnen, was an der eigenen Arbeit neu, besonders oder zumindest anders ist. Wenn man hingegen Grundlagenwerke oder weiterf�hrende Literatur erw�hnt, wird man den Abschnitt �ber verwandte Arbeiten eher am Ende des Hauptteils plazieren oder im Schluss subsummieren. In jedem Fall gibt man eine �bersicht �ber Arbeiten im selben \textit{Themenkreis} oder aber auch Arbeiten, die �hnliche \textit{Methoden} verwenden. 
\subsection{Der Anhang} Man kann in den Anhang jene erg�nzenden Abschnitte ausgliedern, die zwar interessant sein m�gen, aber mit dem roten Faden kaum zu tun haben. Dar�berhinaus finden sich im Anhang oft folgende Abschnitte:
\begin{description}
	\item[Listings:] Von Listings ist grunds�tzlich abzuraten, au�er es handelt sich um sehr kurze, aber f�r die Arbeit sehr essentielle Teile eines Listings. Das k�nnte zum Beispiel ein wichtiger Abschnitt im Firmware-Programm einer selbst gebauten Hardware sein. Ansonsten sind Klassen- und Sequenzdiagramme in UML\footnote{Unified Modeling Language}- Notation eher geeignet, das Verst�ndnis zu erleichtern. Bei Diplomarbeiten sind auf Wunsch des Betreuers die Quell-Codes in elektronischer Form abzugeben. Bei ver�ffentlichten Arbeiten kann man auf Web-Seiten verweisen, f�r den vorliegenden Artikel etwa \citeweblink{merk}.
	\item[Diagramme:] Gesammelte Darstellung von Diagrammen (zum Beispiel Syntaxdiagramme), die im Text erst nach und nach pr�sentiert wurden.
	\item[Glossar:] (gr.-lat.: Erkl�rungsw�rterbuch) Zusammenfassung aller in der Einleitung oder im Text definierten spezifischen Ausdr�cke. Keine Ausdr�cke, die f�r die Zielgruppe der Arbeit selbstverst�ndlich sind! Auch Abk�rzungen k�nnen in diesem Rahmen zusammengefasst werden. Ob Glossar oder Abk�rzungsverzeichnis an den Beginn oder das Ende einer wissenschaftlichen Arbeit geh�ren, ist oft eine Streitfrage. Wenn jedoch alle im Text neu eingef�hrten Begriffe und Abk�rzungen sofort erl�utert werden, dann ist es letztendlich egal, ob das geschlossene Verzeichnis dann vorne oder hinten steht. Der Autor dieses Artikels vertritt die Ansicht, dass Verzeichnisse eher am Ende zusammengefasst werden sollten. 
	\item[Verzeichnisse] von Tabellen, Abbildungen oder Beispielen k�nnen bei k�rzeren Arbeiten den Index ersetzen. Diese Verzeichnisse sollten dann aber auch am Ende der Arbeit angeordnet werden, also dort, wo �blicherweise der Index steht.
	\item[Index:] Der Index, auch Stichwortverzeichnis genannt, ist zwar nicht unbedingt notwendig, wird aber umso empfehlenswerter, je l�nger die Arbeit ist.
	\item[Literatur:] Ein Literaturverzeichnis ist dann unbedingt notwendig, wenn man eine auch noch so kleine Quelle benutzt hat. Schlie�lich zollt man damit den Riesen Anerkennung, auf deren Schultern die eigene Arbeit ruht. Das Literaturverzeichnis soll so genau und ausf�hrlich sein, dass mit der enthaltenen Information die Literaturstelle vom Leser aufgefunden werden kann. Dar�berhinaus ist eine alphabetische Ordnung nach Autorennamen sinnvoll, die Formatierung sollte in etwa der des Literaturverzeichnisses in diesem Artikel entsprechen: ``Autor: \textit{Titel}. Verlag, Erscheinungsort, Erscheinungsjahr". Bei Zeitschriften und Tagungsb�nden analog (siehe Seite \pageref{lit}). Web-Verweise sind von gedruckter Literatur zu unterscheiden, weil sich einerseits das Ziel einer solchen Referenz sp�ter �ndern kann und weil andererseits viele Web-Verweise nicht mit derselben Sorgfalt redigiert und begutachtet wurden, wie gedruckte Literatur. Eventuell sind stabile Verzeichnisadressen einer exakten URL\footnote{Uniform Resource Locator} vorzuziehen (siehe Seite \pageref{wlit}).
\end{description}
\section{Sprachliche Richtlinien}
\label{sr}
In dem Artikel \cite{Tro92} werden etliche sinnvolle Hinweise f�r die sprachliche Korrektheit speziell technischer Arbeiten gegeben. Ansonsten sind in Zweifelsf�llen das �sterreichische W�rterbuch, der Duden oder ein Fremdw�rterlexikon zu bem�hen. Einige besonders wichtige Gedanken sollen aber an dieser Stelle herausgegriffen werden. \\
Es ist nicht notwendig, allgemein �bliche Begriffe, wie Computer, Cursor, Compiler oder Bus 5Uniform Resource Locator einzudeutschen (Rechenmaschine, Schn�rkel, �bersetzer, Datensammelschiene), weil darunter die Verst�ndlichkeit leidet. Allerdings werden diese Begriffe dann im deutschen Text wie Fremdw�rter verwendet und die deutsche Grammatik angewendet (Gro�schreibung, Fallbildung, Artikel, etc). Wo es hingegen wirklich nicht notwendig ist, englische W�rter zu verwenden, sollte man bei den deutschen bleiben. Wer etwa ``searchen" statt ``suchen" im deutschen Text verwendet, sollte ernsthaft erw�gen, seine Arbeit ganz in englischer Sprache zu verfassen. Wie auch immer man sich entscheidet, sollte man jedoch von einem Wort im weiteren Text \textit{entweder} die englische \textit{oder} die deutsche Fassung verwenden, nicht aber beide! \\
Ein besonderes Problem stellt dabei die Koppelung englischer und deutscher Begriffe zu einem Wort dar: Wenn eine unmittelbare Kombination unvermeidlich ist, dann gelten die gleichen Regeln wie in der deutschen Sprache: Trennung durch Divis\footnote{Trennungsstrich, Bindestrich}, erstes und letztes Wort gro� geschrieben, ebenso Substantive, Adjektive, Verben und Adverbien auch innerhalb des zusammengesetzten Wortes. Im �brigen sind solche Kombinationen nach M�glichkeit zu vermeiden, da ohne sie der Text leichter und fl�ssiger zu lesen ist. Dies kann man durch Verwendung rein englischer oder rein deutscher Ausdr�cke erreichen. Oder aber auch durch Umstellung der W�rter, gegebenenfalls unter Zuhilfenahme von Anf�hrungsstrichen. Tabelle \ref{t1} auf Seite \pageref{t1} zeigt einige solcher Kombinationen sowie M�glichkeiten zur Vermeidung derselben. \\
Noch eine abschlie�ende Bemerkung zum Dezimalzeichen: Nach DIN und ISO haben wir in Europa das \textit{Komma} als Dezimalzeichen und nicht den Punkt. Somit hei�t es richtig 0,1 und \textit{nicht} 0.1. Der Punkt als Dezimalzeichen ist allenfalls dann erlaubt, wenn der Ausschnitt aus einem Listing exakt wiedergegeben wird, da es sich dann ja um eine Computer-Sprache und nicht um die deutsche Sprache handelt. 
\section{Besonderheiten spezieller Arbeiten}
Wenn man ein Buch schreiben will, kommt man nicht umhin, die oft sehr detaillierten Vorgaben durch den Verlag zu befolgen. Daher an dieser Stelle nur eine kurze Bemerkung zum Aufbau: Die Kurzfassung befindet sich auf der R�ckseite des Buches, wobei der Aspekt der Werbung in den Vordergrund tritt. Am urspr�nglichen Platz der Kurzfassung hingegen wird ein Vorwort untergebracht, welches oft auch die Danksagung enth�lt. Der Index darf bei einem Buch nicht fehlen. Ansonsten gelten aber auch f�r B�cher die in diesem Artikel zusammengefassten Gestaltungsregeln.
\subsection{Diplomarbeiten}
\begin{table}
\begin{tabular}{|llll|}
\hline
falsch & ung�nstig & besser & eingedeutscht \footnotemark \\
\hline
Instructionschlange & Instruction-Schlange & Instruction Queue & Befehlsschlange\\
Befehlsbuffer & Befehls-Buffer & Instruction Buffer & Befehlspuffer\\
Transceiverbaustein & Transceiver-Baustein & Transceiver Component & Sende-Empf�ngerbaustein\\
Shiftbefehl & Shift-Befehl & Befehl ``shift" & Verschiebebefehl\\
Queueverwaltung & Queue-Verwaltung & Queue-Management & Warteschlangenverwaltung\\
&&Verwaltung der Queue & \\
Latchupeffekt & Latch-up-Effekt & -- & -- \\
\hline
\end{tabular}
\caption{Aul�sung englisch-deutscher Wortkombinationen.}
\label{t1}
\end{table}
\footnotetext{Ist mit Vorsicht anzuwenden, da die englischen Ausdr�cke oft viel gebr�uchlicher sind.}

\begin{figure}
\centering
	\includegraphics{DA_title}
	\caption{Das vorgegebene Deckblatt.}
	\label{f1}
\end{figure}
Das Deckblatt ist vom jeweiligen Dekanat vorgegeben, Abbildung \ref{f1} zeigt ein typisches Beispiel. \\
Die Kurzfassung sollte eine ganze Seite nicht �uberschreiten. Dies darf aber nicht mittels Schriftgr��e und Zeilenabstand erreicht werden, sondern durch inhaltliche Straffung. Meistens ist die Kurzfassung unabh�ngig von der f�r die Arbeit selbst gew�hlten Sprache sowohl in Englisch als auch in Deutsch abzufassen. \\
Der Inhalt der Diplomarbeit sollte mit dem Betreuer abgesprochen werden. Da viele Diplomarbeiten aus einem theoretischen und einem praktischen Teil bestehen, wird sich das auch in der Gliederung der schriftlichen Ausarbeitung widerspiegeln. Dabei sollte auch die Beschreibung des praktischen Teils nicht zu kurz kommen. \\
Der Gesamtumfang einer Diplomarbeit sollte 80 - 100 Seiten (ohne Anhang) nicht �berschreiten. Dem zugrunde liegt i.Allg. die Annahme einer 11 oder 12 Punkt\footnote{Schriftgr��en werden �blicherweise in pt (=Punkte) angegeben, wobei ein Punkt ca. 0,35mm entspricht.} gro�en Schrift mit einem 1,2-fachen Zeilenabstand, wie er in der Satztechnik �blich ist. Weitere spezifische Formatvorgaben k�nnen beim Betreuer erfragt werden. Die in diesem Artikel verwendete zweispaltige Formatierung ist f�r Diplomarbeiten ungeeignet. \\
Dar�ber hinaus sollte w�hrend der Diplomarbeit mit dem Betreuer ein enger Kontakt gepflegt werden. Dies dient vor allem dem Schutz des Diplomanden, weil damit verhindert werden kann, dass sich die Arbeit in einer g�nzlich falschen Richtung verliert; dies w�rde n�mlich viele Stunden vergeblich investierter Arbeitszeit bedeuten \dots \\
\subsection{Referate und Seminararbeiten}
Seminararbeiten berichten selten �ber eine neue Idee, vielmehr fassen sie zu einem Thema vorhandene Literatur zusammen. Daher bekommt der Abschnitt �ber verwandte Arbeiten besondere Bedeutung, er nimmt einen betr�chtlichen Teil des Hauptteils ein. Hier ist es besonders wichtig, die Artikel nicht einfach aufzuz�hlen, sondern zu vergleichen, Gemeinsamkeiten und Unterschiede herauszuarbeiten und Querverbindungen herzustellen. Dabei kommt vor allem einer klaren und strukturierten Klassifizierung der Artikel eine besondere Bedeutung zu. Man sollte sich nicht davon in die Irre leiten lassen, dass viele Autoren dazu tendieren, Unterschiede zu fr�heren Arbeiten hervorzuheben. Oft kommt es sogar vor, dass andere Arbeiten zum Thema als schlecht dargestellt werden. Solche Bewertungen sollte man stets selbst nachpr�fen und nicht kritiklos �bernehmen. \\
Wird eine Seminararbeit als Referat pr�sentiert, so ist auf den Unterschied zwischen einer schriftlichen Ausarbeitung und einer m�ndlichen Pr�sentation zu achten: Als Folien k�nnen nicht einfach Kopien der schriftlichen Ausarbeitung herangezogen werden, ebensowenig darf aus der Seminararbeit einfach vorgelesen werden. Vielmehr gelten f�r Pr�sentationen ganz eigene Regeln. Besonders wichtig ist die Einhaltung der vorgegebenen Rededauer, f�r erg�nzende Informationen hat man in der schriftlichen Seminararbeit Platz, die etwas umfangreicher sein darf als der Vortrag selbst.Besonders empfohlen sei an dieser Stelle der Besuch von Pr�sentationstechnik-Seminaren, denn gutes Pr�sentieren kann man nur durch �bung lernen! 
\subsection{Kurze Artikel}
\label{ka}
Besonders kurze Artikel k�nnen aufgrund ihrer �bersichtlichkeit auf einige Elemente verzichten: Sie besitzen oft kein eigenes Deckblatt und k�nnen auch auf Inhaltsverzeichnis und Index verzichten. Sie sind zumeist nur in Abschnitte gegliedert und nicht mehr in Kapitel, die einzelnen Abschnitte enthalten keine eigenen Einleitungen oder Zusammenfassungen, sondern k�nnen sich voll auf den Inhalt konzentrieren. Ansonsten sollten aber gerade kurze Artikel \textit{besonders} klar gegliedert sein, da sie die Information in sehr kompakter Form vermitteln m�ssen. \\
Als Formatierungsrichtlinie kann \textit{dieser} Artikel selbst dienen: Bei Verwendung der 10-Punkt-Schrift erleichtert die zweispaltige Ausf�hrung das Lesen, da das Auge beim Zeilensprung �ber die Spaltenbreite weniger leicht die Zeile verliert als bei einem Sprung �ber die volle Breite der Seite.
\section{Verwandte Arbeiten}
Der Artikel \cite{Ert93} beschreibt den Aufbau speziell von kurzen Artikeln und Seminararbeiten. In \cite{pug91} sind vor allem Tips f�r das Einsenden an Konferenzen und Zeitschriften enthalten. In vielen Zeitschriften ist immer wieder eine Anleitung f�r Autoren enthalten, zum Beispiel \cite{mar91}. F�r Praktikanten und Diplomanden ist auch \cite{bundy84} sehr empfehlenswert.
\section{Das Wichtigste noch einmal in K�rze}
Um in der Informationsflut �berleben zu k�nnen, muss eine wissenschaftliche Arbeit einen standardisierten Aufbau haben. Neben einem aussagekr�ftigen Titel geh�rt dazu insbesonders eine straffe, klare Kurzfassung, die mit besonderer Sorgfalt verfasst werden muss; die Kurzfassung liefert die Motivation f�r die Besch�ftigung mit der Arbeit! Die klassische, dreigeteilte Gliederung (Einleitung - Hauptteil - Schluss) bew�hrt sich meistens: Die Einleitung bietet mit der Aufgabenstellung den Einstieg ins Thema; bei ihr beginnt der rote Faden, der sich durch die gesamte Arbeit zieht und in der Zusammenfassung endet. Ein Ausblick auf zuk�nftige Arbeiten, eine Aufz�hlung der verwandten Arbeiten sowie ein ausf�hrliches Literaturverzeichnis sollten niemals fehlen. Beachtet man diese Regeln, so wird die Arbeit beim Leser zumindest in die engereWahl genommen werden. Ob sie sich dann auch bew�hrt, h�ngt aber vom Inhalt ab!
\addtocounter{chapter}{1}
\pagestyle{empty}
\newpage
\pagestyle{fancyplain}
\addcontentsline{toc}{chapter}{\protect\numberline{\thechapter}
{\normalsize{Guidlines for writing scientif documents}}}
\includepdf[pages=-,frame=false,fitpaper=false,openright=false,offset=0 0,pagecommand={\thispagestyle{plain}}]{GuidelineRefSemarbDiplDiss_Eng.pdf}

%%%%%%%%%%%%%%%%%%%%%%%%%%%%%%%%%%%%%%%%%%%%%%%
% Appendices
%%%%%%%%%%%%%%%%%%%%%%%%%%%%%%%%%%%%%%%%%%%%%%%

% defines epmty pages except pagenumber in the footer
% generagtes the List of Figures (extension: *.lof)
%\listoffigures
% generates the List of Tables (extension: *.lot)
%\listoftables

% generates the List of Examples and adds it to the toc. If you make use of
% the example-environment uncomment the following lines.
%\listof{example}
%{List of Examples
% \protect
% \addcontentsline{toc}{chapter}{List of Examples}
%}
\pagestyle{empty}
\newpage
\pagestyle{fancyplain}

% generates the aux-Files with citations
% \nocite{*}
% this reads the bbl files and makes the bibliography
\newpage
\setlength{\bibsep}{0.0pt}
\ifisDiss
\renewcommand\bibname{Literature}
\addcontentsline{toc}{chapter}{Literature}
\else
\ifisDipE
\renewcommand\bibname{Literature}
\addcontentsline{toc}{chapter}{Literature}
\else
\ifisDipD
\renewcommand\bibname{Wissenschaftliche Literatur}
\addcontentsline{toc}{chapter}{Wissenschaftliche Literatur}
\else
\ifisBach
\renewcommand\bibname{Wissenschaftliche Literatur}
\addcontentsline{toc}{chapter}{Wissenschaftliche Literatur}
\else
\renewcommand\bibname{Wissenschaftliche Literatur}
\addcontentsline{toc}{chapter}{Wissenschaftliche Literatur}
\fi
\fi
\fi
\fi
\label{lit}
\bibliography{literature}
\bibliographystyle{alphadin}
% the style of bibliography
% start a new page

\newpage
\ifisDiss
\addcontentsline{toc}{chapter}{Internet References}
\else
\ifisDipE
\addcontentsline{toc}{chapter}{Internet References}
\else
\ifisDipD
\addcontentsline{toc}{chapter}{Internet Referenzen}
\else
\ifisBach
\addcontentsline{toc}{chapter}{Internet Referenzen}
\else
\addcontentsline{toc}{chapter}{Internet Referenzen}
\fi
\fi
\fi
\fi
\label{wlit}
\bibliographyweblink{weblinks}
\bibliographystyleweblink{abbrv}
\newpage
% make a new toc entry

% inserts the index % not wanted here
%\printindex
% insert curriculum vitae
\ifisDiss
\chapter*{Curriculum Vitae}
%

\fi
\end{document}
