%%%%%%%%%%%%%%%%%%%%%%%%%%%%%%%%%%%%%%%%%%%%%%%%%%%%%%%%%%%%%%%%%%%%%%%
%                                                                     %
%   Institut f�r Computertechnik, Gusshausstra�e 27-29, 1040 Wien     %
%                                                                     %
% Basisdokument f�r wissenschaftliche Arbeiten am ICT im pdf-    			%
% Format; entsprechenden Schalter unten setzen (\isDiss, etc.)        %
%                                                                     %
%	Basiert auf der Dokumentklasse report.                        			%
%                                                                     %
% Autor: Dietmar Bruckner                                             %
% Version: $Revision: 1.2 $                                           %
% Datum: $Date: 2009/11/25 $				                                  %
%                                                                     %
% --- ICT_Theses_pdf.tex ---  	                                      %
%                                                                     %
%%%%%%%%%%%%%%%%%%%%%%%%%%%%%%%%%%%%%%%%%%%%%%%%%%%%%%%%%%%%%%%%%%%%%%%
% Some options of this template requres the usage of pdfltex.         %
% For the ease of use, please call the makepdf.bat from skripte dir   % 
%%%%%%%%%%%%%%%%%%%%%%%%%%%%%%%%%%%%%%%%%%%%%%%%%%%%%%%%%%%%%%%%%%%%%%%

%%%%%%%%%%%%%%%%%%%%%%%%%%%%%%%%%%%%%%%%%%%%%%%%%%
% Main document for dissertation
%%%%%%%%%%%%%%%%%%%%%%%%%%%%%%%%%%%%%%%%%%%%%%%%%%

%%%%%%%%%%%%%%%%%%%%%%%%%%%%%%%%%%%%%%%%%%%%%%%%%%
% 1.PACKAGES
%%%%%%%%%%%%%%%%%%%%%%%%%%%%%%%%%%%%%%%%%%%%%%%%%%

% defines thesis as report (oneside, A4 with 11pt fontsize)
\documentclass[11pt,twoside,a4paper]{ICTthesis}
% formats the text accourding the set language
\usepackage[english]{babel}
% generates indices with the "\index" command
\usepackage{makeidx}
% enables import of graphics. We use pdflatex here so do the pdf optimisation.
%\usepackage[dvips]{graphicx}
\usepackage[pdftex]{graphicx}
\usepackage{pdfpages}
% includes floating objects like tables and figures.
\usepackage{float}
% for generating subfigures with ohne indented captions
\usepackage[hang]{subfigure}
% redefines and smartens captions of figures and tables (indentation, smaller and boldface)
\usepackage[hang,small,bf]{caption}
% enables tabstops and the numeration of lines
\usepackage{moreverb}
% enables user defined header and footer lines (former "fancyheadings")
\usepackage{fancyhdr}
% Some smart mathematical stuff
\usepackage{amsmath}
% Package for rotating several objects
\usepackage{rotating}
\usepackage{natbib}
\usepackage{epsf}
\usepackage{dsfont}
%\usepackage[usenames]{color}
\usepackage[algochapter, boxruled, vlined]{algorithm2e}
%Activating and setting of character protruding - if you like
%\usepackage[activate,DVIoutput]{pdfcprot}
% If you really need special chars...
\usepackage[latin1]{inputenc}
% Hyperlinks
\usepackage[colorlinks,hyperindex,plainpages=false,%
pdftitle={Master thesis: Sample Title},%
pdfauthor={Authors name},%
pdfsubject={Master thesis},%
pdfkeywords={keyword},%
pdfpagelabels,%
pagebackref,%
bookmarksopen=false%
]{hyperref}
% For the two different reference lists ...
\usepackage{multibib}
\usepackage{multicol}


%%%%%%%%%%%%%%%%%%%%%%%%%%%%%%%%%%%%%%%%%%%%%%%%%%
% 2.Settings
%%%%%%%%%%%%%%%%%%%%%%%%%%%%%%%%%%%%%%%%%%%%%%%%%%

\newif\ifisDiss
\newif\ifisDipE
\newif\ifisDipD
\newif\ifisBach

%VERTIEFUNG/PRAKTIKUM/REFERAT: set all to false;
%and correct respective lines in "`Titlepage_all.tex"'

\isDissfalse
\isDipEfalse
\isDipDfalse
\isBachtrue

\ifisDiss
\newcites{weblink}{Internet References}
\else
\ifisDipE
\newcites{weblink}{Internet References}
\else
\ifisDipD
\newcites{weblink}{Internet Referenzen}
\else
\ifisBach
\newcites{weblink}{Internet Referenzen}
\else
\newcites{weblink}{Internet Referenzen}
\fi
\fi
\fi
\fi
% redifine the paragraph command.
\makeatletter
\renewcommand\paragraph{\@startsection{paragraph}{4}{\z@}%
                                    {3.25ex \@plus1ex \@minus.2ex}%
                                    {0.3em} %-1em}%
                                    {\normalfont\normalsize\bfseries}}

\renewcommand\subparagraph{\@startsection{subparagraph}{5}{\parindent}%
                                       {3.25ex \@plus1ex \@minus .2ex}%
                                       {-1em}%
                                      {\normalfont\normalsize\bfseries}}
\makeatother

%Enables numbers at subsubsections without inserting them into the toc.
\setcounter{secnumdepth}{3}

% generates the index (command for the subprocessor)
\makeindex

% default path to your pictures
\graphicspath{{pictures/}}

% Counter for the maximum number of "Floatobjects" at the beginning of the page.
\setcounter{topnumber}{2}
% Redefines the maximum area which floats my consume at the beginning of the page.
\def\topfraction{.8}
% Counter for the maximum numbers of floats at the end of the page
\setcounter{bottomnumber}{2}
% Redefines the maximum area which floats my consume at the end of the page.
\def\bottomfraction{.5}
% Maximal number of floats per page
\setcounter{totalnumber}{8}
% minimal amount of text per page
\def\textfraction{.2}
% Redefinition: minimal amount of floats in percent per floatpage.
\def\floatpagefraction{.6}
% no indentation at paragraphs
\setlength{\parindent}{0pt}

% part of the caption package: extra 20pts left and right of captions.
%\setlength{\captionmargin}{20pt}

% sets the page layout
\setlength{\oddsidemargin}{4mm}
\setlength{\evensidemargin}{-6mm}
\setlength{\textwidth}{162mm} 
\setlength{\textheight}{230mm}
\setlength{\topmargin}{-5mm}
%\addtolength{\headsep}{12pt}

%part of the "float" Packages:
\floatstyle{plain}
% define a new floating object
\floatname{example}{Example}

\newfloat{example}{hbtp}{loe}[chapter]
\floatplacement{figure}{hbt}
\floatplacement{table}{htb}

% enables a "\dollar" command (returns $)
\newcommand{\dollar}{\char36}

% Script for abbreviations
% defines a new environment with one arguement
\newenvironment{bfscript}[1] {
 % defines as list
 \begin{list}
 % No labelmarks!
 {}
 {\settowidth{\labelwidth}{\small #1}
  % sets the left margin to 0 because there is no labelmark
  \setlength{\leftmargin}{\labelwidth}
  % add labelsep (0, no labelmark) to the margin
  \addtolength{\leftmargin}{\labelsep}
  % Separation of paragraphs in one topic
  \parsep 0.0ex plus 0.2ex minus 0.2ex
  % Separation of two topics
  \itemsep -0.3ex
  % sets the label to: small and fills with whitspace to the text
  \renewcommand{\makelabel}[1]{\small ##1\hfill}}}
 {\end{list}
}

% PDF-Settings
\def\pdfBorderAttrs{/Border [0 0 0] } % No border arround Links
\pdfcompresslevel=9
\hypersetup{colorlinks,linkcolor=blue,filecolor=red,urlcolor=black,citecolor=blue}

%%%%%%%%%%%%%%%%%%%%%%%%%%%%%%%%%%%%%%%%%%%%%%%%%%
% 3.HYPENATION
%%%%%%%%%%%%%%%%%%%%%%%%%%%%%%%%%%%%%%%%%%%%%%%%%%

% enter special rules here!
\hyphenation{gleich-zeitig para-meter}

%%%%%%%%%%%%%%%%%%%%%%%%%%%%%%%%%%%%%%%%%%%%%%%%%
% 4.Begin of the real document
%%%%%%%%%%%%%%%%%%%%%%%%%%%%%%%%%%%%%%%%%%%%%%%%%%

\begin{document}
% reads the new commands


\newcommand{\mat}[1]{\ensuremath{\mathbf #1}}
\newcommand{\set}[1]{\ensuremath{\mathbf #1}}
\newcommand{\cset}[1]{\ensuremath{\mathbf{\mathcal #1}}}
\renewcommand{\vec}[1]{\ensuremath{\mathbf #1}}

\newcommand{\nth}[1]{\ensuremath{#1^\mathrm{th}}}
\newcommand{\fst}[1]{\ensuremath{#1^\mathrm{st}}}
\newcommand{\snd}[1]{\ensuremath{#1^\mathrm{nd}}}

\newcommand{\argmax}[1]{\ensuremath{\arg\hspace{-0.4ex}\max_{\hspace*{-3.0ex}#1}}}
\newcommand{\argmin}[1]{\ensuremath{\arg\min_{\hspace*{-4.0ex}#1}}}

\newcommand{\argmaxi}[1]{\ensuremath{\arg\hspace{-0.4ex}\max_{#1}}}
\newcommand{\argmini}[1]{\ensuremath{\arg\hspace{-0.4ex}\min_{#1}}}

% \newcommand{\argmin}[1]{\ensuremath{\begin{array}[t]{c} \arg \min \\
% \vspace*{-0.1ex} #1 \end{array}}}

\newcommand{\NP}{\ensuremath{\mathcal{NP}}}
\newcommand{\PP}{\ensuremath{\mathcal{P}}}
\newcommand{\e}[2]{\ensuremath{\{#1,#2\}}}
\newcommand{\tup}[1]{\ensuremath{\langle#1\rangle}}
\newcommand{\bigO}[1]{\ensuremath{\mathcal{O}\left(#1\right)}}

\newcommand{\trans}[1]{\ensuremath{{#1}^\top}}
\newcommand{\diag}[1]{\ensuremath{\mathrm{diag}\left(#1\right)}}

\newcommand{\eq}[1]{equation \ref{#1}}
\newcommand{\Eq}[1]{equation \ref{#1}}
\newcommand{\fig}[1]{figure \ref{#1}}
\newcommand{\Fig}[1]{figure \ref{#1}}
\newcommand{\chap}[1]{chapter \ref{#1}}
\newcommand{\Chap}[1]{chapter \ref{#1}}
\newcommand{\sect}[1]{section \ref{#1}}
\newcommand{\Sect}[1]{section \ref{#1}}

\newcommand{\bydefn}{\ensuremath{\stackrel{\bigtriangleup}{=}}}
\newcommand{\elmat}[2]{\ensuremath{#1 \odot #2}}

\newcommand{\prune}[1]{\ensuremath{\mathrm{prune}\left(#1\right)}}

\newcommand{\labelfig}[2]{\parbox[b]{0.2in}{\Large#1\normalsize\vspace{#2}}}
% \newcommand{\labelfig}[1]{\parbox[b]{0.2in}{#1\vspace{1.8in}}}

% \newcommand{\emptyset}{\ensuremath{\O}}

\renewcommand{\Re}{\mathbb{R}}

\newcommand{\incfig}[3]{\ifx\pdfoutput\undefined
                          \epsfig{#1.eps,#2,#3}
                        \else
                          \epsfig{#1.eps,#2,#3}
                        \fi}
          
                         

% inserts the Titlepage
\pdfbookmark{Titlepage}{title}
%%%%%%%%%%%%%%%%%%%%%%%%%%%%%%%%%%%%%%%%%%%%%%%%%%%%%%%%%%%%%%%%%%%%%%%
%                                                                     %
%   Institut f�r Computertechnik, Gusshausstra�e 27-29, 1040 Wien     %
%                                                                     %
% Basisdokument f�r wissenschaftliche Arbeiten am ICT im pdf-   			%
% Format                                                              %
%                                                                     %
%	Basiert auf die Dokumentklasse report.                        			%
%                                                                     %
% Autor: Dietmar Bruckner                                             %
% Version:                                                            %
% Datum: 28. Aug. 2005                                                %
%                                                                     %
% --- Titelseite.tex ---                                              % 
%                                                                     %
%%%%%%%%%%%%%%%%%%%%%%%%%%%%%%%%%%%%%%%%%%%%%%%%%%%%%%%%%%%%%%%%%%%%%%%
% This document defines the normalized preface of the document.       %
% Please adjust only:                                                 %
% *) any names                                                        %
% *) Abstract                                                         %
% *) Kurzfassung                                                      %
% *) Acknowledgement                                                  %
% *) possible Abbreviations                                           %
%%%%%%%%%%%%%%%%%%%%%%%%%%%%%%%%%%%%%%%%%%%%%%%%%%%%%%%%%%%%%%%%%%%%%%%

\pagestyle{empty}
\pagenumbering{Roman}
\begin{titlepage}
\large
\begin{center}
\ifisDiss
DISSERTATION\\\vfill
{\LARGE\bf Your Title}
\\
[5mm]
Your Subtitle
\else
\ifisDipE
DIPLOMA THESIS\\\vfill
{\LARGE\bf Your Title}
\\
[5mm]
Your Subtitle
\else
\ifisDipD
DIPLOMARBEIT\\\vfill
{\LARGE\bf Titel}
\\
[5mm]
ev. Untertitel
\else
\ifisBach
BACHELORARBEIT\\\vfill
{\LARGE\bf Titel}
\\
[5mm]
ev. Untertitel
\else
VERTIEFUNG/PRAKTIKUM/REFERAT\\\vfill
{\LARGE\bf Titel}
\fi
\fi
\fi
\fi
\\
\vfill
\ifisDiss
Submitted at the
Faculty of Electrical Engineering and Information Technology, TU Wien, Vienna
in partial fulfillment of the requirements for the degree of\\
Doktor der technischen Wissenschaften (equals Ph.D.)\\
\vfill
under supervision of\\\vfill
Full name and degrees of your Supervisor\\
Institut number: 384\\
Institute of Computer Technology\\\vfill
and\\\vfill
Full name and degrees of your Supervisor\\
Institut number: 384\\
Institute of Computer Technology\\\vfill
by\\\vfill
Your full name, Firstname first, then surname\\
Matr.Nr. xxxxxxx\\
single lined adress: street number, zip city\\\vfill
\end{center}
Date\hfill\hrulefill
\else
\ifisDipE
Submitted at the
Faculty of Electrical Engineering and Information Technology, Vienna University of Technology\\
in partial fulfillment of the requirements for the degree of\\
Diplom-Ingenieur (equals Master of Sciences) \\
\vfill
under supervision of\\\vfill
Full name and degrees of your Supervisor Prof.\\
Full name and degrees of your Supervisor\\\vfill
by\\\vfill
Your full name, Firstname first, then surname\\
Matr.Nr. xxxxxxx\\
single lined adress: street number, zip city\\\vfill
\end{center}
Date\hfill\hrulefill
\else
\ifisDipD
ausgef�hrt zur Erlangung des akademischen Grades \\
eines Diplom-Ingenieurs unter der Leitung von\\
\vfill
Name und akad. Grad des Prof.\\
Name und akad. Grad des Betreuers\\\vfill
am\\\vfill
{\Large\bf Institut f�r Computertechnik (E384)}\\
der Technischen Universit�t Wien
\vfill
durch
\vfill
Name\\
Matr.Nr. xxxxxxx\\
Wohnadresse\\\vfill
\end{center}
Wien, am xxx\hfill\hrulefill
\else
\ifisBach
ausgef�hrt zur Erlangung des akademischen Grades \\
eines Bachelors unter der Leitung von\\
\vfill
Name und akad. Grad des Prof.\\
Name und akad. Grad des Betreuers\\\vfill
am\\\vfill
{\Large\bf Institut f�r Computertechnik (E384)}\\
der Technischen Universit�t Wien
\vfill
durch
\vfill
Name\\
Matr.Nr. xxxxxxx\\
Wohnadresse\\\vfill
\end{center}
Wien, am xxx\hfill\hrulefill
\else
ausgef�hrt als Projektdokumentation/Referatsarbeit/Praktikum unter der Leitung von\\
\vfill
Name und akad. Grad des Prof.\\
Name und akad. Grad des Betreuers\\\vfill
am\\\vfill
{\Large\bf Institut f�r Computertechnik (E384)}\\
der Technischen Universit�t Wien
\vfill
durch
\vfill
Name\\
Matr.Nr. xxxxxxx\\
Wohnadresse\\\vfill
\end{center}
Wien, am xxx\hfill\hrulefill
\fi
\fi
\fi
\fi
\end{titlepage}
%

\newpage
\pagestyle{plain}
\null\vfil
\begin{center}\bf Kurzfassung\end{center}
Deutsche Kurzfassung\dots\\
\par\vfil
\ifisDiss
\newpage
\fi
\null\vfil
\begin{center}\bf Abstract\end{center}
short version of your thesis \dots
\par\vfil\null
%
\ifisDiss
\newpage
\null\vfil
\begin{center}\bf Vorwort/Preface\end{center}
\par\vfil\null
%
%
\fi
\newpage
\null\vfil
\ifisDiss
\begin{center}\bf Acknowledgements (optional)\end{center}
Thank you!
\else
\ifisDipE
\begin{center}\bf Acknowledgements (optional)\end{center}
Thank you!
\else
\ifisDipD
\begin{center}\bf Danksagung (optional)\end{center}
Danke!
\else
\ifisBach
\begin{center}\bf Danksagung (optional)\end{center}
Danke!
\else
\begin{center}\bf Danksagung (optional)\end{center}
Danke!
\fi
\fi
\fi
\fi
\par\vfil\null
%
\newpage
%
\ifisDiss
\renewcommand{\contentsname}{Table of Contents}
\else
\ifisDipE
\renewcommand{\contentsname}{Table of Contents}
\else
\ifisDipD
\renewcommand{\contentsname}{Inhaltsverzeichnis}
\else
\ifisBach
\renewcommand{\contentsname}{Inhaltsverzeichnis}
\else
\renewcommand{\contentsname}{Inhaltsverzeichnis}
\fi
\fi
\fi
\fi
\tableofcontents
%
\ifisDiss
\chapter*{Abbreviations}
%\thispagestyle{empty}
\markboth{Abbreviations}{}
%
\begin{bfscript}{Somethings}
% enter your common and not so comman abbrev. here
\item[ABB] Abbreviation
\item[ABB] Abbreviation
\item[ABB] Abbreviation
\item[ABB] Abbreviation
\item[ABB] Abbreviation
\end{bfscript}
\else
\ifisDipE
\chapter*{Abbreviations}
%\thispagestyle{empty}
\markboth{Abbreviations}{}
%
\begin{bfscript}{Somethings}
% enter your common and not so comman abbrev. here
\item[ABB] Abbreviation
\item[ABB] Abbreviation
\item[ABB] Abbreviation
\item[ABB] Abbreviation
\item[ABB] Abbreviation
\end{bfscript}
\else
\ifisDipD
\chapter*{Abk�rzungen}
%\thispagestyle{empty}
\markboth{Abk�rzungen}{}
%
\begin{bfscript}{Irgendwas}
% enter your common and not so comman abbrev. here
\item[ABB] Also Bin Baff
\item[ABB] Also Bin Baff
\item[ABB] Also Bin Baff
\item[ABB] Also Bin Baff
\item[ABB] Also Bin Baff
\end{bfscript}
\else
\ifisBach
\chapter*{Abk�rzungen}
%\thispagestyle{empty}
\markboth{Abk�rzungen}{}
%
\begin{bfscript}{Irgendwas}
% enter your common and not so comman abbrev. here
\item[ABB] Also Bin Baff
\item[ABB] Also Bin Baff
\item[ABB] Also Bin Baff
\item[ABB] Also Bin Baff
\item[ABB] Also Bin Baff
\end{bfscript}
\else
\chapter*{Abk�rzungen}
%\thispagestyle{empty}
\markboth{Abk�rzungen}{}
%
\begin{bfscript}{Irgendwas}
% enter your common and not so comman abbrev. here
\item[ABB] Also Bin Baff
\item[ABB] Also Bin Baff
\item[ABB] Also Bin Baff
\item[ABB] Also Bin Baff
\item[ABB] Also Bin Baff
\end{bfscript}
\fi
\fi
\fi
\fi
%
\newpage
\pagenumbering{arabic}


% defines the pagestyle of the document to fancy package
\pagestyle{fancy}

% redefinitions of the plain style for the first pages of chapters.
\fancypagestyle{plain}{%
\fancyhf{}%
\fancyfoot[C]{\thepage}%
\renewcommand{\headrulewidth}{0pt}%
\renewcommand{\footrulewidth}{0pt}}

% twosided Head: {Chaptertitle                      }{                      Chaptertitle}
\renewcommand{\chaptermark}[1]{\markboth{}{\slshape\small{#1}}}
\renewcommand{\sectionmark}[1]{}

% After the preface use arabic numbers for pagecounting
% is already defined in titlepage.tex

% vertical separaiont of paragraphs
\setlength{\parskip}{1.5ex plus 1ex minus 0.5ex}

% include your chapters here.
% Do not use number words in the filenames, you may run into trouble reconfiguring your
% document otherwise!
%%%%%%%%%%%%%%%%%%%%%%%%%%%%%%%%%%%%%%%%%%%%%%%%%%%%%%%%%%%%%%%%%%%%%%%%%%%%%%%%
%                                                                              %
%	File:     abstract.tex                                                     %
%   Document: XXX	                                                           %
%   Author:   Freismuth David                                                  %
%	Date:	  22.JUN.2018                                                      %
%   Content:  Contains the abstract section of the Bachelor thesis.            %
%                                                                              %
%%%%%%%%%%%%%%%%%%%%%%%%%%%%%%%%%%%%%%%%%%%%%%%%%%%%%%%%%%%%%%%%%%%%%%%%%%%%%%%%

%%%%%%%%%%%%%%%%%%%%%%%%%%%%%%%%%%%%%%%%%%%%%%%%%%%%%%%%%%%%%%%%%%%%%%%%%%%%%%%%
\paragraph{Abstract}
Due to increasing size and complexity of modern hardware designs, the challenge
of identifying a piece of design becomes increasingly difficult. This is especially
true, if no documentation is available. This factor has a direct impact on the 
time that is needed to get familiar with a design. In extreme cases, the design
is rendered useless for the user. A hint on what hardware category the design 
belongs to, would accelerate the process of familiarization.
This work considers, if it is possible to categorize hardware designs, that are
given as Hardware Description Language, on basis of their structure. 
The elaborated algorithm is able to categorize a given design in X seconds, with
an accuracy of S.

%%%%%%%%%%%%%%%%%%%%%%%%%%%%%%%%%%%%%%%%%%%%%%%%%%%%%%%%%%%%%%%%%%%%%%%%%%%%%%%%
\paragraph{Kurzfassung}
Mit steigender Größe und Komplexität von modernen Hardware Designs, wird es 
zusehends herausfordernder die Funktion des desselben zu identifizieren. Vor
allem trifft dies zu, wenn keine Dokumentationen zum Design verfügbar sind. Dieser 
Umstand wirkt sich unmittelbar in einer erhöhten Einarbeitungszeit aus. In Extremfällen 
muss der Anwender das Design wegen Unbrauchbarkeit verwerfen. Ein Hinweis darauf welcher
Hardware Kategorie das Design angehört, würde den Einarbeitungsprozess beschleunigen.
Diese Arbeit untersucht, ob es möglich ist Hardware Designs, die als Hardware
Description Language vorliegen, anhand ihres strukturellen Aufbaus zu klassifiziern, 
und in Kategorien einzuteilen. 
Mithilfe des erarbeiteten Algorithmus ist es möglich ein Design innerhalb von X 
Sekunden, mit einer Sicherheit von Y zu klassifiziern. 

\addtocounter{chapter}{1}
\pagestyle{empty}
\newpage
\pagestyle{fancyplain}
\addcontentsline{toc}{chapter}{\protect\numberline{\thechapter}
{\normalsize{Guidlines for writing scientif documents}}}
\includepdf[pages=-,frame=false,fitpaper=false,openright=false,offset=0 0,pagecommand={\thispagestyle{plain}}]{GuidelineRefSemarbDiplDiss_Eng.pdf}

%%%%%%%%%%%%%%%%%%%%%%%%%%%%%%%%%%%%%%%%%%%%%%%
% Appendices
%%%%%%%%%%%%%%%%%%%%%%%%%%%%%%%%%%%%%%%%%%%%%%%

% defines epmty pages except pagenumber in the footer
% generagtes the List of Figures (extension: *.lof)
%\listoffigures
% generates the List of Tables (extension: *.lot)
%\listoftables

% generates the List of Examples and adds it to the toc. If you make use of
% the example-environment uncomment the following lines.
%\listof{example}
%{List of Examples
% \protect
% \addcontentsline{toc}{chapter}{List of Examples}
%}
\pagestyle{empty}
\newpage
\pagestyle{fancyplain}

% generates the aux-Files with citations
% \nocite{*}
% this reads the bbl files and makes the bibliography
\newpage
\setlength{\bibsep}{0.0pt}
\ifisDiss
\renewcommand\bibname{Literature}
\addcontentsline{toc}{chapter}{Literature}
\else
\ifisDipE
\renewcommand\bibname{Literature}
\addcontentsline{toc}{chapter}{Literature}
\else
\ifisDipD
\renewcommand\bibname{Wissenschaftliche Literatur}
\addcontentsline{toc}{chapter}{Wissenschaftliche Literatur}
\else
\ifisBach
\renewcommand\bibname{Wissenschaftliche Literatur}
\addcontentsline{toc}{chapter}{Wissenschaftliche Literatur}
\else
\renewcommand\bibname{Wissenschaftliche Literatur}
\addcontentsline{toc}{chapter}{Wissenschaftliche Literatur}
\fi
\fi
\fi
\fi
\label{lit}
\bibliography{literature}
\bibliographystyle{alphadin}
% the style of bibliography
% start a new page

\newpage
\ifisDiss
\addcontentsline{toc}{chapter}{Internet References}
\else
\ifisDipE
\addcontentsline{toc}{chapter}{Internet References}
\else
\ifisDipD
\addcontentsline{toc}{chapter}{Internet Referenzen}
\else
\ifisBach
\addcontentsline{toc}{chapter}{Internet Referenzen}
\else
\addcontentsline{toc}{chapter}{Internet Referenzen}
\fi
\fi
\fi
\fi
\label{wlit}
\bibliographyweblink{weblinks}
\bibliographystyleweblink{abbrv}
\newpage
% make a new toc entry

% inserts the index % not wanted here
%\printindex
% insert curriculum vitae
\ifisDiss
\chapter*{Curriculum Vitae}
%

\fi
\end{document}
