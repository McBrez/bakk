%%%%%%%%%%%%%%%%%%%%%%%%%%%%%%%%%%%%%%%%%%%%%%%%%%%%%%%%%%%%%%%%%%%%%%%%%%%%%%%%
%                                                                              %
%	File:     abstract.tex                                                     %
%   Document: XXX	                                                           %
%   Author:   Freismuth David                                                  %
%	Date:	  22.JUN.2018                                                      %
%   Content:  Contains the abstract section of the Bachelor thesis.            %
%                                                                              %
%%%%%%%%%%%%%%%%%%%%%%%%%%%%%%%%%%%%%%%%%%%%%%%%%%%%%%%%%%%%%%%%%%%%%%%%%%%%%%%%

%%%%%%%%%%%%%%%%%%%%%%%%%%%%%%%%%%%%%%%%%%%%%%%%%%%%%%%%%%%%%%%%%%%%%%%%%%%%%%%%
\paragraph{Abstract}
Due to increasing size and complexity of modern hardware designs, the challenge
of identifying a piece of design becomes increasingly difficult. This is especially
true, if no documentation is available. This factor has a direct impact on the 
time that is needed to get familiar with a design. In extreme cases, the design
is rendered useless for the user. A hint on what hardware category the design 
belongs to, would accelerate the process of familiarization.
This work considers, if it is possible to categorize hardware designs, that are
given as Hardware Description Language, on basis of their structure. 
The elaborated algorithm is able to categorize a given design in X seconds, with
an accuracy of S.

%%%%%%%%%%%%%%%%%%%%%%%%%%%%%%%%%%%%%%%%%%%%%%%%%%%%%%%%%%%%%%%%%%%%%%%%%%%%%%%%
\paragraph{Kurzfassung}
Mit steigender Größe und Komplexität von modernen Hardware Designs, wird es 
zusehends herausfordernder die Funktion des desselben zu identifizieren. Vor
allem trifft dies zu, wenn keine Dokumentationen zum Design verfügbar sind. Dieser 
Umstand wirkt sich unmittelbar in einer erhöhten Einarbeitungszeit aus. In Extremfällen 
muss der Anwender das Design wegen Unbrauchbarkeit verwerfen. Ein Hinweis darauf welcher
Hardware Kategorie das Design angehört, würde den Einarbeitungsprozess beschleunigen.
Diese Arbeit untersucht, ob es möglich ist Hardware Designs, die als Hardware
Description Language vorliegen, anhand ihres strukturellen Aufbaus zu klassifiziern, 
und in Kategorien einzuteilen. 
Mithilfe des erarbeiteten Algorithmus ist es möglich ein Design innerhalb von X 
Sekunden, mit einer Sicherheit von Y zu klassifiziern. 
