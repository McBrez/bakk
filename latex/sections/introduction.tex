%%%%%%%%%%%%%%%%%%%%%%%%%%%%%%%%%%%%%%%%%%%%%%%%%%%%%%%%%%%%%%%%%%%%%%%%%%%%%%%%
%                                                                              %
%	File:     introduction.tex                                             %
%       Document: XXX	                                                       %
%       Author:   Freismuth David                                              %
%	Date:	  22.JUN.2018                                                  %
%       Content:  Contains the Introduction section of the Bachelor thesis.    %
%                                                                              %
%%%%%%%%%%%%%%%%%%%%%%%%%%%%%%%%%%%%%%%%%%%%%%%%%%%%%%%%%%%%%%%%%%%%%%%%%%%%%%%%

%%%%%%%%%%%%%%%%%%%%%%%%%%%%%%%%%%%%%%%%%%%%%%%%%%%%%%%%%%%%%%%%%%%%%%%%%%%%%%%%
\section{Introduction}
As Embedded Systems find their way in an increasingly wide field of applications,
with growing demands to performance and relieability, the underlying Hardware 
Designs also gain in diversity, complexity and size. Since it is nearly impossible,
even for simple Designs, to determine the function of such, without proper
documentation, a possibility to extract information directly from the Hardware 
Description Language representation of the Design seems to be a welcome helper. 
